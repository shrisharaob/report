% Chapter 1

\chapter{Internal representation of space} % Main chapter title

\label{Chapter1} 

\lhead{Chapter 1. \emph{Internal Representation of space}} % This is for the header on each page

%----------------------------------------------------------------------------------------
Spatial representation in animals is essential for their survival since the animal has to search for food and mates, which demands considerable spatial processing to succeed. Representation of space requires that the geometric relationship between objects are encoded with a convenient metric. The relevant objects and landmarks in the environment can thus be organized in a spatial framework, although the structure of space is independent of the sensory input through which the former is extracted. Spatial information provides the context for adaptive behaviours. Storing spatial relationships over time could be a plausible framework for encoding episodic memory. \\
Egocentric or allocentric representations can be used to encode and store spatial information. Egocentric space corresponds to the representation relative to the viewer's current location. If the viewer moves, the egocentric representation also changes correspondingly. In allocentric representation, the spatial relationships of landmarks and objects in the environment are encoded relative to each other, independent of the viewer. Unless the objects move, the allocentric representation is stable even when the viewer moves. 

\section{Spatial navigation in the the rat}

\subsection{Place cells}
 \emph{Place cells} were first discovered by O'Keefe and Dostrovsky \cite{O'Keefe1971a}. Following this discovery, O'Keefe and Nadel \cite{Street}  conceived of a functional role of hippocampus, they proposed that the hippocampus is the substrate for a cognitive map which is formed while the rat explores its environment. This is known as the \emph{cognitive map theory}. They suggested that the primary function of the hippocampus was to encompass mechanisms that would allow for the existence of the map in the hippcampal network. The cognitive map represents spatial information in allocentric space. So the relative spatial metrics of the landmarks in the incoming stimuli have to be computed. This map can then be utilized for for solving navigational tasks while the various structures of the hippocampus provide the spatial information related to place, head direction, and current view. They also suggested that the map is stored in the hippocampus and it is used in conjunction with information stored in other areas. By extending their theory to humans they postulated that episodic memory derives its basis from similar mechanisms that enable spatial processing in animals. Leison studies corroborate their claims. The fornix fibres are one of the predominant fibre tracts into the hippocampus . Lesions in the track lead to partial deficit in spatial learning in rats, but cue learning is preserved.   \\

Place cells are the  pyramidal cells in the rat and mice hippocampus. Place cells derive their name form the fact that their discharge is strongly correlated with the position of the animal in the environment. In a given unchanging environment, each cell has a preferred location called the place field where the cell fires maximally.  The cells that are active in one environment may not participate in another. The total number of cells active in a given environment is proposed to be 25-30 \% providing a sparse representation which increases storage capacity and reduces interference between representations\cite{Marr2007, Wilson1993}. Recent $Ca^{2+}$ imaging of CA1 pyramidal cells are consistent with this account \cite{Ziv2013}.
Place cells are found in the CA1 and CA3 regions of the dorsal hippocampus. They have also been found in the ventral hippocampus.
Place cell activity in response to environmental manipulations have been intensively studied. The trajectories traversed by the rat determine the directionality of the place cell \cite{Save1998}. If the place cell is indeed coding for a specific location, then its firing must be independent of  head direction. The angular firing distribution is found to pass this test in the open field and the modulation of firing by head direction in a maze can be explained as a result of non-uniform sampling of head directions at every location \cite{Muller1994}. It is possible that in the open field the place cells achieve omni-directionality by associative binding of local views \cite{Sharp1991}. Hence, omni-directional might develop through andom exploration rather than directed exploration in a maze.
\begin{itemize}
\item place fields can be also associated with non-visual cues 
Muller-Kubie 1987 and O'Keefe, Speakman 1987 - 
\item Place fields continued to persist after the visual land marks were removed  \cite{Kubie1987}
\item removing some cues did not disrupt the place fields but removing all cues did Pivo 1985 -
Quirk, Muller and Kubie 1990 - place fields persisted in the dark when the animals are placed with light and the light turned off.\\
O'keefe, Recce 1993 and Skaags 1996 - spike fired precesses with theta phase as the animal traverses the place field, also occurs in 2D\\
Mehta, Barenes and Mc Naughton1997 - pfs shift backwards after repeated traversals along the same trajectory 
Place cell firing is also correlated with speed, direction, and turning angle  and stage of the task \\
Markus 1995 - pfs are task sensitive, pfs rapidly changed when the task changed to random foraging and search for food at the corners of a diamond\\
Skaags McNaughton 1996 - found significant correlated temporal bias during SWRs in the SWS after exploration on a linear track.\\
extra hippocampal cells have also been reported, but show different characteristics the hippocampal place cells\\
Wilson MccNaughton 1993 - place cells recorded with tetrode shower fewer subfields\\

\item Huxter, Csicsvari 2008, Nat.Neur; Theta phase-specific
codes for two-dimensional position, trajectory and head-
ing in the hippocampus.

\end{itemize}

\section{Head Direction cells}
Often when one is given a map and asked to find his way to a certain goal location, one starts by orienting himself with the help of a compass or cues such as the sun or the starts. Thus for successful spatial navigation and route planning, there must be neural system which provides a stable reference direction in an environment. A subset of cells in the postsubiculum  are found to be sensitive to head direction \cite{Taube1990}. The head direction system is perhaps wired to implement a continuous ring attractor which receives vestibular and visual inputs. These inputs are integrated to generate a stable activity distribution along the perceived head direction. If there is considerable mismatch between the inputs will result in the reference direction being reset along the orientation implied by the stronger input.  [lab meeting paper]

\section{Grid Cells}
\cite{Hafting2005}

\section{Path integration}
Path integration, also referred to as dead reckoning is one of the strategies that can be used for spatial navigation. It keeps track of the the current location with respect to a reference point in the current reference frame. The ability to path integrate is vital for the survival of animals, it provides a means of computing the home location after the animal has traveled away from home location in search of food.
Self motion information (speed and acceleration) and heading direction are essential information that need to be available for path integration. If the speed and heading direction is known at an arbitrary location, then the future location can be readily computed. Unless the ideothetic information is noisy, a perfect path integration system is capable of maintaining location information with respect to a reference point. Any noise that enters the measurements of speed and direction will lead to accumulative error in the predicted location. This increasing error can be corrected if the system has access to location information with respect to external landmarks. In animals this is the sensory input that is constantly available. Both self motion cues and sensory information when used together provide a better estimation of the current location. Interesting experiments can be designed in virtual reality by introducing appreciable conflict between sensory input and ideothetic information. 
%Interesting phenomena occur in the place field characteristics when there is appreciable conflict between sensory input and the path integration estimation.

\subsection{Multichart architechture}
\label{pathIntegration}
One of theories proposes that the synaptic connections of the CA3 pyramidal cells are pre-configured to represent a large number of two-dimensional surfaces \cite{Samsonovich1997}. These two dimensional surfaces are continuous attractors with localized activity patterns as their stable states. A simple way to understand how the weights are assigned is to imagine a plane, then placing a randomly chosen subset of cells on this plane. Then the synaptic strength between any proximally located pair of cells is given by a two dimensional gaussian function of the distance between them. The cells further away receive inhibitory connections. This synaptic connectivity scheme produces 2 dimensional quasi-continuous attractors. The imaginary arrangement of a population of place cells on an abstract plane such that each  cell is fixed at a location where it shows maximum firing activity is termed to as a chart. When this chart is bound to an environment, then each cell shows peak firing at the appropriate location in the physical location to which it has been mapped. Their model proposes that multiple charts are available which could be potentially used to represent several environments or the same environment in differing contexts. These charts exhibit no significant correlations. A chart is referred to as the active chart if the activity of cells in that chart appears to be localized at a specific location thus forming a place field. The distribution of activity can remain localized even in the absence if external stimuli since it is a stable state of the network. The active chart would then be the current reference frame used for spatial processing. Other charts would would display rand activity as dictated by the synaptic connections. \\
In their model place cells connect to a PI(path integration system) which sends back asymmetric projections to cells along the direction of motion of the animal. Thus the localized activity pattern in the active chart follows the actual movement of the animal.

%----------------------------------------------------------------------
% SECTION REMAPPING
%----------------------------------------------------------------------
\section{remapping}
\label{remapping}
When certain features of the environment are varied by small amounts the place cell firing characteristics are altered \cite{Kubie1987}. The firing rates are observed to show drastic changes. Some place cells change their place fields to new locations while others completely vanish and new fields emerge. This is often referred to as \emph{remapping}, resulting from the changes in the spatial information of the environment. Thus distinct representations are formed for differing environments and even similar environment with apparently minor modifications. \\
Storage of memory requires that the each unique memory engram produces decorrelated patterns of activity. If the memories share some common features, it will lead activation of very similar activity patterns and the distinction might be lost. This is often referred to as interference and is a possible mechanism of forgetting. To keep memories distinct a decorrelation operation of the overlapping memories must be performed to so as to produce orthogonal representations before storage. Remapping might belie the underlying decorrelation of overlapping inputs that occurs when memories are encoded. \\
Two types of remapping can be identified. Rate remapping occurs when the location of the place fields remain essential the same while the firing rate changes. Global remapping occurs when the cells arbitrarily change their firing rates and develop new place fields. These categories of remapping signal different kinds of environmental manipulations. Rate remapping was observed when the location was unchanging while the color and shape of enclosures were changed. Global remapping was induced in identical enclosures at different location \cite{Leutgeb2005a}. Changing cue configurations also produce global remapping \cite{Leutgeb2005a}.
Rate remapping probably occurs when only the relevant non-spatial features change while the spatial structure is preserved. Thus the spatial information content in the population is identical for rate remapped representations since it is only the non-spatial component of the stimuli is varying. In the $Ca^{2+}$ imaging study \cite{Ziv2013}, $15-25 \% $ of CA1 cells were found to consistently encoding for space over weeks. Rate remapping could then also be the short time scale characteristic of the ensemble encoding scheme. This would enable the encoding of other non-spatial information as episodes in an unchanging spatial setting. The only overlapping patterns the ensemble code would be coding for the same environments. \\
Global remapping is almost always induced when the location is changed. The magnitude of dissimilarity between the relevant environmental features dictates the likelihood of global and rate remapping. \\
The earlier experiments made discrete changes in the environment. The naturally occurring stimuli are usually continuous. The hippocampal network then has to generate dynamically varying patterns to encode the continuously changing stimuli. In a study by Leutgeb \cite{Leutgeb2005},the flexible enclosures were slowly morphed from square to circle with several intermediate stages. The animals were initially familiarized with both square and circular enclosures so as to allow for stable representations of each to be learnt. Rate remapping was observed for the gradual changes from square to circle.  [transitions in the representation]\\
In another study \cite{Wills2005}, the square and circular enclosures were designed to induce global remapping. There was abrupt transition to a learnt representations based on the similarity of the current environment to the previously acquired one. \\
The CA3 pyramidal cells are observed to produce significantly distinct patterns in response to apparently minor changes in the environment. Since the CA3 network is also thought to function as an auto-associative memory, this would be an advantageous feature increasing the network capacity. The encoding scheme has to produce non-redundant codes for effective performance of the auto-associative network. The Dentage Gyrus input seems to be critical for pattern separation in the CA3. The Dentate Gyrus granulae cells are hypothesized to perform the computational operation of orthogonalization of the input patterns before feeding them to the CA3 network. The sparse yet effective connectivity between the granulae cells and the CA3 pyramidal cells could provide the critical architecture.\\ The grid cells in the Medial entorhinal cortex(MEC) show little change in the grid parameters whenever the environmental change produces rate remapping in CA3. But when global remapping occurs in CA3, the grid cells show coherent shift and rotation of their fields. The time course of grid realignment and remapping seem to be closely follow each other \cite{Fyhn2007}. Perhaps the grid realignment in the MEC contributes to global remapping in the CA3. Since the CA3 also back projects to the MEC, it is also not known whether the realignment of the grids occurs first in the MEC and then leads to CA3 remapping or vice versa.\\ 
Remapping the is less pronounced in CA1 in comparison to CA3 \cite{Leutgeb2005a, Leutgeb2004}. It also possible that the CA1 representation changes over a longer time scale. The interneuron network assists the formation of new cell assemblies and supression of old ones \cite{Dupret2013}

%---------------------------------------------------------------------
% SECTION REPLAY
%---------------------------------------------------------------------
\section{Replay}
\label{replay}
...
Marr \cite{Marr2007} proposed that long term storage and classification of information as the functional role of neocortex. He also proposed that the neocortex would be required to be trained during sleep to classify overlapping pieces of information. The hippocampus is thought to be a temporary storage for memory( ). Then the hippocampus has to meet the computational demands of rapid encoding and later recapitulation of salient memory traces for transfer to long term storage. Marr also suggested that the anatomically the short term storage structure must have the means to communicate with every other part of the brain. The Ento-rhinal cortex has reciprocal connections to [...........]




