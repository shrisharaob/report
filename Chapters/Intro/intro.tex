\chapter{Introduction} 

\label{intro}

One \todo{test comment} of the striking features of the brain is its ability to generate self organized activity without the necessary interaction with its of external environment. Although this fact by itself does not imply any functional role played by the self generated activity, it certainly rules out the possibility of completely understanding the brain and its subsystems in a simple stimulus - response paradigm. The task of deducing the mechanisms and computations performed in the sub-structures and  the brain as a whole, would greatly benefit from an approach that examines the brain from an inside-out perspective. This approach would attempt to investigate the internal states that extensively influence the behaviour of the system under scrutiny. The internal states constrain the behaviour of the system when it is interacting with its surroundings and also when the system is isolated from its environment. The internal states might actually be essential for performing the computations required for the appropriate functioning of the system. \\

The oscillations observed in large scale recordings of the brain are one of the chief examples of internal  activity. These oscillatory states are thought to organize the computations at the appropriate time scale. An analogy can be draw between these oscillations and the clock in a microprocessor which organizes the fetching and execution of a sequence of instructions. Although this analogy cannot be taken too far, since the brain is not a sequential processing unit. A set of clocks which tick at different rates seems beneficial for organizing a massive distributed processing system such as the brain. \\

The \emph{Hippocampus} is a structure of the brain that is the focal point of all converging sensory inputs. It presents an excellent window into the internal network dynamics of the brain, far away from the primary somatosensory areas. The hippocampus has a simple and well organised anatomical composition, which provides the opportunity to conduct large scale recordings with fine time resolution. Internal activity is not restricted only to the cortex, the electro-physiology of the hippocampus has revealed several internally self generated patterns. Although during sleep the sensory input is at a minimum, these recordings show a lot of interesting patterns. Some of which are thought to be essential for memory consolidation and spatial processing.  
The hippocampus has come under intensive behaviour combined electro-physiological studies in the context of spatial processing. All these experimental studies address the overarching questions of how the hippocampal network forms spatial representations. To what extent this representation is influenced by the internal dynamics and how this internal activity modulates or is modulated by the incoming sensory stimuli. The examination of these experimental works has revealed a lot about the mechanisms of not just spatial processing, but also has lead to theories postulating general computational architecture of the brain. This study attempts to shed light on general questions related to the degree to which internal states constrain and steer the activity at the network level when it is interacting with its environment and during states with little or no external input. \\
The following sections are intended to be a brief review of the relevant literature and concepts to set the context for this study.
\section{Anatomy}
The hippocampal formation has been identified to be composed of several regions based on their cytoarchitecture. These are Dentate Gyrus, CA3, CA1, Subiculum, Entorhinal cortex, pre and para Subiculum. Sensory information from different modalities converges onto the hippocampus via the Entorihnal cortex and the reciprocal divergent connections from the hippocampus to the neocortex also pass through the EC. EC sends projections to DG and CA1 through the perforant pathway. DG sends mossy fibers to CA3. CA3 projects to CA1 through the Schaffer collaterals. CA3 has massive reccurent connections with other pyramidal cells in the same area. CA1 projects to EC, via projections to deep layers. 
% Chapter 1

\section{Internal representation of space} % Main chapter title

\label{space} 

\lhead{Chapter 1. \emph{Internal Representation of space}} % This is for the header on each page

%----------------------------------------------------------------------------------------
Spatial representation in animals is vital for their survival since they have to search for food and mates, this makes considerable demands in terms of spatial processing for success. Representation of space requires that the geometric relationship between objects in the environment are encoded using a convenient metric. The relevant objects and landmarks in the environment can thus be organized in a spatial framework, although the structure of space is independent of the sensory input through which the former is extracted. Spatial information provides the context for adaptive behaviours. Storing spatial relationships over time could be a plausible framework for encoding episodic memory. \\
Egocentric or allocentric representations can be used to encode and store spatial information. Egocentric space corresponds to the representation relative to the viewer's current location. If the viewer moves, the egocentric representation also changes correspondingly. In allocentric representation, the spatial relationships of landmarks and objects in the environment are encoded relative to each other, independent of the viewer. Unless the objects move, the allocentric representation is stable even when the viewer moves. Hippocampus is a structure that has been implicated to play an important role in spatial processing and episodic memory. In this section I will review some of the prominent discoveries made in experiments of spatial navigation conducted on rats and mice.

\subsection{Spatial navigation in the the rat}

\subsubsection{Place cells}
\label{placeCells}
 \emph{Place cells} were first discovered by O'Keefe and Dostrovsky \cite{O'Keefe1971a}. Following this discovery, O'Keefe and Nadel \cite{Street}  conceived of a functional role for the hippocampus, they proposed that the hippocampus is the substrate for a cognitive map which is formed while the rat explores its environment. This is known as the \emph{cognitive map theory}. They suggested that the primary function of the hippocampus was to encompass mechanisms that would enable the learning and storage of the map in the hippcampal network. The cognitive map represents spatial information in allocentric space. So the relative spatial metrics of the landmarks in the incoming stimuli have to be computed. This map can then be utilized for for solving navigational tasks while the various structures of the hippocampus provide the spatial information related to place, head direction, and current view. They also suggested that the map is stored in the hippocampus and it is used in conjunction with information stored in other areas. By extending their theory to humans they postulated that episodic memory derives its basis from similar mechanisms that enable spatial processing in animals. Leison studies corroborate their claims. The fornix fibers are one of the predominant fiber tracts into the hippocampus . Lesions in the track lead to partial deficit in spatial learning in rats, but cue learning is preserved.   \\

Place cells are the  pyramidal cells in the rat and mice hippocampus. Place cells derive their name form the fact that their discharge is strongly correlated with the position of the animal in the environment. In a given unchanging environment, each cell has a preferred location called the place field where the cell fires maximally.  The cells that are active in one environment may not participate in another. The total number of cells active in a given environment is proposed to be 25-30 \% providing a sparse representation which increases storage capacity and reduces interference between representations\cite{Marr2007, Wilson1993}. Recent $Ca^{2+}$ imaging of CA1 pyramidal cells are consistent with this account \cite{Ziv2013}.
Place cells are found in the CA1 and CA3 regions of the dorsal hippocampus. They have also been found in the ventral hippocampus.\\
Place cell activity in response to environmental manipulations have been intensively studied. The trajectories traversed by the rat determine the directionality of the place cell \cite{Save1998}. If the place cell is indeed coding for a specific location, then its firing must be independent of  head direction. The angular firing distribution is found to pass this test in the open field and any modulation of firing by head direction in a maze can be explained as a result of non-uniform sampling of head directions at every location \cite{Muller1994}. It is possible that in the open field the place cells achieve omni-directionality by associative binding of local views \cite{Sharp1991}. Hence, the place cells might develop omni-directionality through random exploration rather than directed exploration in a linear or radial maze.\\
The place cell spiking times are known to show reliable relationship with respect to the ongoing theta rhythm when the rat is in motion. The firing of the place cells show theta phase precession in one dimensional tracks \cite{O'Keefe1993} as well as in unconstrained open field \cite{Skaggs1996c} [one more ref]. When the animal enters the place field of a cell, the cell fires at late phase of theta and as the rate traverses through the place field it fires at earlier phases of successive theta cycles. The cell fires maximally at the trough of theta when the rat is at location corresponding to the peak of the place field tuning curve. Thus the theta phase of the spike times is correlated with the distance the animal has travelled along the place field. This is often referred to as temporal coding \cite{Huxter2003}. When several cells are recorded, the cells with overlapping place fields fire at different phases of theta, which is suggested as an encoding strategy for compressing sequences in time. This has the advantage that these sequences are arranged in the time scale of LTP to influence the synaptic connectivity between the cells participating in the sequence. Thus providing a mechanism for rapid encoding of current experience. \\
The place cells are not associated only with visual cues. The place fields are observed to persist even after the removal of visual landmarks \cite{Kubie1987}. In an experiment where the animals were placed with lights on and subsequently turned off, most place fields did not disappear in the dark \cite{Muller2008}. So the place fields are derived by integrating both visual and non-visual cues. \\ 

\begin{itemize}
Mehta, Barenes and Mc Naughton1997 - pfs shift backwards after repeated traversals along the same trajectory 
Place cell firing is also correlated with speed, direction, and turning angle  and stage of the task \\
Markus 1995 - pfs are task sensitive, pfs rapidly changed when the task changed to random foraging and search for food at the corners of a diamond\\

extra hippocampal cells have also been reported, but show different characteristics the hippocampal place cells\\
Wilson MccNaughton 1993 - place cells recorded with tetrode shower fewer subfields\\

\item Huxter, Csicsvari 2008, Nat.Neur; Theta phase-specific
codes for two-dimensional position, trajectory and head-
ing in the hippocampus.

\end{itemize}
 phase precession
 
\subsection{Head Direction cells}
Often when someone is given a map and asked to find his way to a certain goal location, one starts by orienting oneself with the help of a compass or cues such as the sun or the starts. Thus for successful spatial navigation and route planning with the cognitive map, there must be a neural system which provides a stable reference direction in an environment. A subset of cells in the postsubiculum  are found to be sensitive to head direction \cite{Taube1990}. The head direction tracking can be achieved through angular integration. The head direction system is perhaps wired to implement a continuous ring attractor which receives vestibular and visual inputs. These inputs are integrated to generate a stable activity distribution along the perceived head direction. If there is considerable mismatch between the inputs will result in the reference direction being reset along the orientation implied by the stronger input \cite{Valerio2012}.

\subsection{Grid Cells}
The neurons in the Medial Entorihnal cortex (MEC) are also found to be spatially tuned. These cells fire at regular distances along the animals trajectory. Their tuning curves show gird like structure spanning the whole environment. The vertices of the grid form equilateral triangles \cite{Hafting2005}. The grid cell population comprises grid cells of different scales, orientations and phases. They provide an internally  calibrated scale for spatial computations. How the grid structure emerges is yet to be understood. The hippocampal back projections to MEC are critical for the reliable grid structure \cite{Bonnevie2013}. 

\subsection{Path integration}
\label{pathIntgr}
Path integration, also referred to as dead reckoning is one of the strategies that can be used for spatial navigation. It keeps track of the the current location with respect to a reference point in the current reference frame. The ability to path integrate is vital for the survival of animals, it provides a means of computing the home location after the animal has traveled away from home location in search of food.
Self motion information (speed and acceleration) and heading direction are essential information that need to be available for path integration. If the speed and heading direction is known at an arbitrary location, then the future location can be readily computed. Unless the ideothetic information is noisy, a perfect path integration system is capable of maintaining location information with respect to a reference point. Any noise that enters the measurements of speed and direction will lead to accumulative error in the predicted location. This increasing error can be corrected if the system has access to location information with respect to external landmarks. In animals this is the sensory input that is constantly available. Both self motion cues and sensory information when used together provide a better estimation of the current location. Interesting experiments can be designed in virtual reality by introducing appreciable conflict between sensory input and ideothetic information. 
%Interesting phenomena occur in the place field characteristics when there is appreciable conflict between sensory input and the path integration estimation.

\subsubsection{Multichart architechture}
\label{pathIntegration}
One of theories proposes that the synaptic connections of the CA3 pyramidal cells are pre-configured to represent a large number of two-dimensional surfaces \cite{Samsonovich1997}. The pre-configured architecture is achieved through special synaptic connectivity. Along each surface the synaptic strength between neurons any two neurons is a function of the distance between the peaks of place fields.  %These two dimensional manifolds in the neural space form continuous attractors with localized activity patterns as their stable states. 
A simple way to understand how the weights are assigned is to imagine a plane with randomly chosen subset of cells placed on this plane. Then the synaptic strength between any proximally located pair of cells is given by a two dimensional gaussian function of the distance between them. The cells further away receive inhibitory connections. This synaptic connectivity scheme produces 2 dimensional quasi-continuous attractors. The imaginary arrangement of a population of place cells on an abstract plane such that each  cell is fixed at a location where it shows maximum firing activity is called a \emph{chart}. When this chart is bound to an environment, then each cell shows peak firing at the appropriate location in the physical location to which it has been mapped. Their model proposes that multiple \empth{charts} are available which could be potentially used to represent several environments or the same environment in differing contexts. These \emph{charts} exhibit no significant correlations, thus reducing  interference between representations. A \empth{chart} is referred to as the \emph{active chart} if the activity of cells in that chart appears to be localized at a specific location on the that chart. The active neurons that proximally in the active chart would be allocated random positions that are different on different charts. So the activity distribution on all other charts would appear to be dispersed. The distribution of activity can remain localized even in the absence if external stimuli since it is a stable state of the network. The active chart would then be the current reference frame used for spatial processing. Other charts would would display rand activity as dictated by the synaptic connections. \\
In their model place cells connect to a PI(path integration system) which sends back asymmetric projections to cells along the direction of motion of the animal. Thus the localized activity pattern in the active chart follows the actual movement of the animal.

%----------------------------------------------------------------------
% subsection REMAPPING
%----------------------------------------------------------------------
\subsection{remapping}
\label{remapping}
When certain features of the environment are varied by small amounts the place cell firing characteristics are altered \cite{Kubie1987}. The firing rates are observed to show drastic changes. Some place cells change their place fields to new locations while others completely vanish and new fields emerge. This is often referred to as \emph{remapping}, resulting from the changes in the spatial information of the environment. Thus distinct representations are formed for differing environments and even similar environment with apparently minor modifications. \\
Storage of memory requires that the each unique memory engram produces decorrelated patterns of activity. If the memories share some common features, it will lead activation of very similar activity patterns and the distinction between them might be lost. This is often referred to as interference and is a possible mechanism of forgetting. To keep memories distinct, a decorrelation operation of the overlapping memories must be performed to so as to produce orthogonal representations before storage. Remapping might be a signature of the underlying decorrelation computations occurring while memories are encoded.\\
Two types of remapping can be identified. Rate remapping occurs when the location of the place fields remain essential the same while the firing rates change. Global remapping occurs when the cells arbitrarily change their firing rates and develop new place fields. These categories of remapping signal different kinds of environmental manipulations. Rate remapping was observed when the location was unchanging while the color and shape of enclosures were changed. Global remapping was induced in identical enclosures at different locations \cite{Leutgeb2005a}. Changing cue configurations also produce global remapping \cite{Leutgeb2005a}.
Rate remapping probably occurs when only the relevant non-spatial features change while the spatial structure is preserved. Thus the spatial information content in the population is identical for rate remapped representations since it is only the non-spatial component of the stimuli is varying. In the $Ca^{2+}$ imaging study \cite{Ziv2013}, $15-25 \% $ of CA1 cells were found to consistently encoding for space over weeks. Rate remapping could then also be the short time scale characteristic of the ensemble encoding scheme where the population of cells recruited are changing. This would enable the encoding of other non-spatial information as episodes in an unchanging spatial setting. The only overlapping patterns in the ensemble code would be coding for the unchanging spatial information. \\
Global remapping is almost always induced when the location is changed. The magnitude of dissimilarity between the relevant environmental features dictates the likelihood of global and rate remapping. \\
The earlier experiments made discrete changes in the environment. The naturally occurring stimuli are usually continuous. The hippocampal network then has to generate dynamically varying patterns to encode the continuously changing stimuli. In a study by Leutgeb \cite{Leutgeb2005}, flexible enclosures were gradually morphed from a square shape to circle with several intermediate stages. The animals were initially familiarized with both square and circular enclosures so as to allow for stable representations of each to be learnt. Rate remapping was observed for the gradual changes from square to circle.  [transitions in the representation] [teleportation]\\
In another study \cite{Wills2005}, the square and circular enclosures were designed to induce global remapping. There was abrupt transition to a learnt representations based on the similarity of the current environment to the previously acquired one. \\
The CA3 pyramidal cells are observed to produce significantly distinct patterns in response to apparently minor changes in the environment. Since the CA3 network is also thought to function as an auto-associative memory, this would be an advantageous feature increasing the network capacity. The encoding scheme has to produce non-redundant codes for effective performance of the auto-associative network. The Dentage Gyrus (DG) input seems to be critical for pattern separation in the CA3. The DG granulae cells are hypothesized to perform the computational operation of orthogonalization of the input patterns before feeding them to the CA3 network. The sparse yet effective connectivity between the granulae cells and the CA3 pyramidal cells could provide the critical architecture.\\ The grid cells in the Medial entorhinal cortex(MEC) show little change in the grid parameters whenever the environmental change produces rate remapping in CA3. But when global remapping occurs in CA3, the grid cells show coherent shift and rotation of their fields. The time course of grid realignment and remapping seem to be closely follow each other \cite{Fyhn2007}. Perhaps the grid realignment in the MEC contributes to global remapping in the CA3. Since the CA3 also back projects to the MEC, it is also not known whether the realignment of the grids occurs first in the MEC and then leads to CA3 remapping or vice versa. Remapping the is less pronounced in CA1 in comparison to CA3 \cite{Leutgeb2005a, Leutgeb2004}. It also possible that the CA1 representation changes over a longer time scale. \\ 
The interneuron network assists the formation of new cell assemblies and supression of old ones \cite{Dupret2013}, playing an important role in the emergencce of remapped representations.

%---------------------------------------------------------------------
% subsection REPLAY
%---------------------------------------------------------------------
\subsection{Replay}
\label{replay}
The hippocampus is thought to be a temporary storage for memory \cite{Buzsaki1989, Battaglia2011}. Then the hippocampus has to meet the computational demands of rapid encoding and later recapitulation of salient memory traces for transfer to long term storage. The phenomenon of replay is a possible mechanism through which consolidation might be achieved.  Marr \cite{Marr2007} proposed that long term storage and classification of information as the functional role of neocortex. He also proposed that the neocortex would be required to be trained during sleep to classify overlapping pieces of information. Since the hippocampus has suitable anatomical communication channels via the entorhinal cortex, it is appropriate for short term storage. The Entorhinal cortex in turn has reciprocal connection with other regions. \\

Replay is the phenomena where the cell sequences that are activated along a trajectory are later activated in the same or reverse order when the animal is sleeping or even when the animal is awake during periods of immobility . The synaptic connectivity of the cells participating in a sequence during exploration are strengthened.  The sequences appearing when the animal is exploring the environment have greater propensity to occur during subsequent sleep. Significant correlated temporal bias favouring the same cell sequences to repeat during SWRs in the SWS after exploration on a linear track has been reported \cite{Skaggs1996b}. \\







\section{Dynamcial systems framework}
\label{dynamics}

\lhead{Chapter 1. \emph{Dynamical systems}} 

%----------------------------------------------------------------------------------------
%	subsection 1
%----------------------------------------------------------------------------------------

Dynamical systems theory deals with time varying systems. It dwells on questions of how the behaviour of systems evolve over time. Non-linear dynamics is a powerful analytic tool for studying complex systems. A network of neurons can be formalized as a system of differential equations, lending them amenable to be studied in the same framework. A network with recurrent connections is a dynamical system. Some models of hippocampal phenomena derive insights form dynamical systems approach. \\
There is the long standing idea that interesting abstract properties emerge in a population of interacting objects obeying certain local rules. This framework might prove to be useful in developing further insights into the mechanisms that give rise to emergent cognitive faculties.\\
\\
Differential equations provide a convenient and straight forward technique to formalize and study dynamical systems. The system under study is formulated by specifying governing equations capturing the dynamics of the system. The variables in these equations are called state variables. Solving these equations given initial conditions, we can predict the future states of the system. One soon realizes that this is no trivial matter for complex systems. Some systems are inherently unpredictable event though the governing equations are deterministic. Higher order differential equations can be converted to a system if first order differential equations. The system of differential equations can be imagined as a vector field in the state space where a vector is assigned to every point in the state space. The vector points in the direction of change of the state variables and its magnitude conveys the rate of change of the state. 
 
\subsection{Basic concepts}

\subsubsection{Fixed points}
When some systems attain certain states, they show no further change provided that there is no external perturbation. These states are called the $fixed points$ of the system i.e where the flow is zero and consequently the state remains fixed.  \st{this figure shows a one dimensional system whose vector field is defined by some arbitrary function $f(x)$ When $f(x)$ is positive the system moves towards $+ \inf$ when $f(x) = 0$, there is no change. These states are called the fixed points of the system.} A fixed point is stable if the system can recover form small perturbations about the fixed point.  An unstable fixed point is one about which there is no restoring force that steers the system back to the fixed point.  A mechanical equivalent of unstable fixed point is an inverted pendulum.

%\st{When there are two or more state variables, different types of fixed points are observed : \\
%\begin{itemize}
%\item star node
%\end{itemize}
%fig- show a two dimensional system where different types of fixed points result when the parameter %$a$ is varied. These are the stable nodes star node , line of fixed points , saddle node.} \\


Usually the behavior of a dynamical system can be predicted from its energy functional, if it exists. It is a scalar function of the state variables. Without external input, any system always tends to evolve to a state with lower energy. The standard picture is to imagine a ball rolling down the hill towards the valley and always ending up in the valley. The energy landscapes can be complex in biological systems. Some systems are highly sensitive to initial conditions. The ball can end up in one of the several valleys depending on where it starts. A simple example is that of bi-stable system which has been used for modelling working memory. One of the states is the resting state, other is with persistent activity which could function as working memory.\\
The stable fixed points of the systems are interesting since if they exist, the transients die out and the system will eventually settle to one of the stable states. These stable states are called attractors of the system. An attractor is the limiting set of states that the system will approach as $t\, \rightarrow \, \infty $. The topology of the attractors can be helpful in predicting the long term behaviour of a system. A point attractor is a single point in the state space towards which all system evolution trajectories converge. A line attractor is a single continuous arrangement of point attractors. Limit cycle attractors give rise to periodic behavior of the systems. Limit cycles cannot occur in linear systems. Biological pattern generators can be understood in terms of limit cycles. Rhythmic spiking neurons have stable limit cycle attractors.  \st{Here is an example of modified HH model... x axis is the membrane voltage , y is teh potassium activation variable, Initially th model neuron is at rest corresponding to a stable point. If a strong pulse of current is injected in the membrane , it will take the neuron to the basin of attraction of the limit cycle and the neuron will produce rhythmic spikes.} \\
\st{point. line attractor If end points of the line attractor are joined, a ring attractor is obtained which is proposed a model of the head direction system.} \\

\subsubsection{Stability analysis}
The stability of fixed points can be analyzed by linearizing the system at the fixed point and then applying linear stability analysis techniques. This gives correct predictions of stability around the fixed point. When linear methods fails, Layapnov stability analysis approach is usually adopted. Lyapnov function provides a generalized energy landscape and conservative estimate of domains of attraction . 
\\
\subsubsection{Bifurcations}
Certain parameters of systems when varied, result in it exhibiting qualitatively new behavior, the system is then said to have undergone a bifurcation. Bifurcations reflect the dependence on parameters. At some critical values of the parameters, the system switches between different regimes of operation. So the phase space can be partitioned into regions with qualitatively distinct behaviors.\\In neural networks, the same network could be switched between different regimes to implement different computations, where neuro-modulators levels might be responsible for switching.\\ Bifurcations are classified as local and global. Global if the bifurcation effects a large portion of the state space. Bifurcations are classified based on how the properties of fixed points change.  
\begin{itemize}
\item \emph{Saddle node bifurcation} occurs when a stable and an unstable fixed points get closer and closer as the parameter is varied , eventually colliding and vanishing. \st{ variable and the fixed points are plotted as the dependent variable. The dotted lines indicate the unstable nodes and the stable mode.}

\item \emph{pitchfork bifurcation} occurs in symmetric systems as the parameter is varied it loses stability and two new fixed points are created. \st{Eg - a beam buckling when overloaded load}

\item \emph{Hopf bifurcation} occurs when a stable spiral loses its stability and a limit cycle is created. \st{model neuron the injected current as the parameter , when ramp current is applied at a critical value the spiking.}

\end{itemize}
The hippocampus exhibits different states when the animal is mobile and exploring its environment. When the animal is immobile or sleeping, the activity is qualitatively different \cite{Buzsaki2011,  Montgomery2008}.
At the network level, this type of transition between different network states could be brought about by the network undergoing bifurcations. 

\subsection{Attractor networks}
Attractor networks have been theoretically shown to have several computational capabilities \cite{Amit1992}, especially for computations in noisy systems.\\
Auto-assosiative memory and map based path integration, functions that are attributed to the hippocampus; have been hypothesized to be implemented in the attractor network of the CA3. Attractor networks have stable patterns as their attractor sets and depending on the initial conditions the network will settle down to one of the stable patterns. Depending on the type of attractor different kinds patterns can be achieved. Point attractors serve to memorize patterns. Localized activity patterns are achieved by local excitation and long range inhibition. If the point attractors are very close to each other, this will result in a continuous attractor \cite{Trappenberg2003}. It is possible to stabilize activity at every point along a continuous attractor. Continuous attractor dynamics has been proposed as the underlying dynamics for path integration (sec. \ref{pathIntgr}, p. \pageref{pathIntgr}). Several other experimentally observed hippocampal phenomena could be explained as a signature of attractor dynamics, where simple feed-forward network explanations fail.\\

\subsubsection{Auto-associative network}
\label{autoasso}
For reliable recall of memories, pattern completion capabilities would be advantageous for a memory system. The recurrent synaptic connectivity of CA3 is suitable for implementing an auto associative network. An auto-associative network memorizes patterns through local learning rules that modify the synaptic strengths. Each pattern produces an energy minima in the energy landscape. Pattern completion occurs because of the recurrent synaptic architecture. When a few neurons of a certain pattern are active, then they activate others because of strong synaptic connectivity strength. CA3 representation is found to be coherent between environments with small changes in cues \cite{Lee2004, Vazdarjanova2004}, which could be explained as an effect of pattern completion. \\
Experiments which observe the phenomenon of \emph{remapping} \cite{Kubie1987, Wills2005, Leutgeb2005} (sec. \ref{remapping}, p. \pageref{remapping})have put to doubt the auto-associative function usually ascribed to  hippocampus \cite{Colgin2010}. Global remapping might be caused when the path integration system coordinates are reset, which may be brought about by changing of location.   

\subsubsection{Models of place selectivity and phase precession}
Some models of place cells are constructed as networks with attractor dynamics. The basic requirement to get place cell like activity is to obtain localized firing patterns, so that the this pattern might then be utilized to encode distinct locations in the physical space. Second requirement is to have a mechanism by which this localized activity can be conveniently updated as the animal traverses in the environment.  \\

Contrary to the view that there is an explicit temporal code (sec. \ref{placeCells} p. \pageref{placeCells}), it has been suggested that the correlation between position and phase of theta could be explained as a consequence of sequential computation occurring within a theta cycle. Across the population of place cells, different phases the theta cycle encode positions offset into either future or past along the rat's trajectory \cite{Itskov2008}. The past and future locations robustly predict the CA1 place cell activity at different theta phases. This indicates that the information content in the cell activity at different phases of theta is actually correlated with the past and future locations of the animal. It was also show that this phenomena is not a direct consequence of phase precession, rather might actually be a causing the observed phase precession. \\

Tsodyks \cite{Tsodyks1996} proposed a model proposing possible mechanisms through with place cell selectivity is achieved through a specific synaptic connectivity between neurons reflecting the distances between their place field peaks in the environment. In the presence of global inhibition this architecture results in attractor dynamics. External input with weak selectivity is sufficient to steer the network into the one of the basins of attraction resulting in stable localized activity. The activity tracks the input peak location as it moves. With the addition of asymmetric synaptic strengths in the direction of motion, phase precession effect is observed. Hence the phase precession could be a manifestation of the inherent asymmetry in the synaptic connections. The inhibitory interneurons in the model network receive theta input which is the simulated input from medial septum to the GABAergic interneurons. The simulation of the network shows that as the rat moves along the track the external excitation drifts through a group of neurons. Due to the asymmetry in the synaptic connections the activity spontaneously propagates forward in every theta cycle. \\  

\subsubsection{Internal sequences in the rat hippocampus}
In the study by Pastalkova \cite{Pastalkova2008a}, multi-unit recordings were obtained form the rat hippocampus in non-sleep state. They report observation of internally self sustained cell assembly sequences. This has been modeled as a network with attractor dynamics. The animals were trained to run in a wheel during the delay period in an alternation task. The CA1 pyramidal neurons were recorded during the delay period. Some of these neuron assemblies were sequentially activated. These sequences were predictive of the time the rat spent in the wheel upto 20 seconds \cite{Itskov2011a}. The sequences were unique for different behavioral choices including the ones which were incorrect choices. Since the location of the rat was stationary, one would expect to see only place cells for that location to be active.  \\  
%[time/distance cells]
It has been proposed that this could be a means by which the networks keeps track of time elapsed. A model was proposed suggesting a possible mechanism for the generation of cell sequences in a network with no strong inputs \cite{Itskov2011a}. The two critical ingredients of the model are adaptive thresholds and Mexican hat synaptic connectivity. In the model, every spike fired by the neuron will result in the increase of its spiking threshold. This threshold then exponentially decays to its default value with a time constant in the order of seconds. Thus a neuron gets increasingly discouraged to fire as its firing rate increases, which will eventually silence the neuron.The Mexican hat type connectivity ensures that the activity remains localized to a small number of neighboring neurons. The symmetry in the synaptic weights is broken by introducing uncorrelated noise in the connectivity matrix. In fact, to ensure that the dynamics displayed by the model is not just a result of the perfect synaptic tuning; the strength of the correlated Mexican hat connectivity was chosen to be weaker as compared to that of the heterogeneous component. This model generates a continuum of bump attractors, for the network, this implies that if the bump were to be moved laterally by some means, it would stabilize in the new position.  The introduction of threshold adaptation will result in the increase of the thresholds of neurons participating in the localized activity. Gradually, on the time scale of seconds (i.e. the time scale of threshold relaxation), this bump will lose its stability as the neurons become quite due high values of threshold. Assuming the input noise levels are low, the heterogeneity in the network connectivity will dictate the next position of the stable bump, which will remain so until it is destabilized again by the same mechanism. The bump in essence is moving away from the neurons with recently updated thresholds. In this setting, activity bump shifts its peak constantly without ever stabilizing at a particular location. Thus the model exhibits self generated sequential activation of cells, captured by the bump constantly moving along continuous trajectories in the state space. The model produces reliable trajectories even with weakly noisy input, provided that it starts with the same initial conditions. Hence, the similar heterogeneity and threshold levels across trials provide identical initial conditions (contexts) which results in reproducible behavior. It was also shown that these sequences can be inherited by succeeding layer of neurons without recurrent connections receiving sparse feed-forward input from the layer with recurrent connectivity. Since the CA1 region does not have recurrent connectivity and the sequences were observed in CA1, it is possible that they are inherited form sequences are generated elsewhere.


\subsubsection{Preplay}
The trajectories taken by an animal are rapidly encoded as sequences of place cells. It has been proposed that this can be accomplished if there exists a pool of sequences ready to be bound to sensory cues. The mechanism of \emph{preplay} has been reported as evidence for this claim  \cite{Dragoi2011, Dragoi2013a}. 

\section{Objectives}
[INCOMPLETE]\st{
This study was aimed at characterizing the spatio-temporal dynamics of the hippocampal network. The place cell activity dynamics across multiple environments were investigated. How do the firing 


Since 2 dimensional environments are closer to the spatial layout in the natural world, the place cells in 2D arena was analyzed in this study. The population activity in the CA3 and CA1 were analysed 


 The place cells in 2D arena was analyzed in this study.}\st{find evidence for the existence of attractors in the hippocampal network. 
search for the computational algorithms used in the hippocampus and its supporting structures for path integration and episodic memory. how well does the data agree with the existing models of path integration and remapping}


%\section{Anatomy}
The hippocampal formation has been identified to be composed of several regions based on their cytoarchitecture. These are Dentate Gyrus, CA3, CA1, Subiculum, Entorhinal cortex, pre and para Subiculum. Sensory information from different modalities converges onto the hippocampus via the Entorihnal cortex and the reciprocal divergent connections from the hippocampus to the neocortex also pass through the EC. EC sends projections to DG and CA1 through the perforant pathway. DG sends mossy fibers to CA3. CA3 projects to CA1 through the Schaffer collaterals. CA3 has massive reccurent connections with other pyramidal cells in the same area. CA1 projects to EC, via projections to deep layers. 
%% Chapter 1

\section{Internal representation of space} % Main chapter title

\label{space} 

\lhead{Chapter 1. \emph{Internal Representation of space}} % This is for the header on each page

%----------------------------------------------------------------------------------------
Spatial representation in animals is vital for their survival since they have to search for food and mates, this makes considerable demands in terms of spatial processing for success. Representation of space requires that the geometric relationship between objects in the environment are encoded using a convenient metric. The relevant objects and landmarks in the environment can thus be organized in a spatial framework, although the structure of space is independent of the sensory input through which the former is extracted. Spatial information provides the context for adaptive behaviours. Storing spatial relationships over time could be a plausible framework for encoding episodic memory. \\
Egocentric or allocentric representations can be used to encode and store spatial information. Egocentric space corresponds to the representation relative to the viewer's current location. If the viewer moves, the egocentric representation also changes correspondingly. In allocentric representation, the spatial relationships of landmarks and objects in the environment are encoded relative to each other, independent of the viewer. Unless the objects move, the allocentric representation is stable even when the viewer moves. Hippocampus is a structure that has been implicated to play an important role in spatial processing and episodic memory. In this section I will review some of the prominent discoveries made in experiments of spatial navigation conducted on rats and mice.

\subsection{Spatial navigation in the the rat}

\subsubsection{Place cells}
\label{placeCells}
 \emph{Place cells} were first discovered by O'Keefe and Dostrovsky \cite{O'Keefe1971a}. Following this discovery, O'Keefe and Nadel \cite{Street}  conceived of a functional role for the hippocampus, they proposed that the hippocampus is the substrate for a cognitive map which is formed while the rat explores its environment. This is known as the \emph{cognitive map theory}. They suggested that the primary function of the hippocampus was to encompass mechanisms that would enable the learning and storage of the map in the hippcampal network. The cognitive map represents spatial information in allocentric space. So the relative spatial metrics of the landmarks in the incoming stimuli have to be computed. This map can then be utilized for for solving navigational tasks while the various structures of the hippocampus provide the spatial information related to place, head direction, and current view. They also suggested that the map is stored in the hippocampus and it is used in conjunction with information stored in other areas. By extending their theory to humans they postulated that episodic memory derives its basis from similar mechanisms that enable spatial processing in animals. Leison studies corroborate their claims. The fornix fibers are one of the predominant fiber tracts into the hippocampus . Lesions in the track lead to partial deficit in spatial learning in rats, but cue learning is preserved.   \\

Place cells are the  pyramidal cells in the rat and mice hippocampus. Place cells derive their name form the fact that their discharge is strongly correlated with the position of the animal in the environment. In a given unchanging environment, each cell has a preferred location called the place field where the cell fires maximally.  The cells that are active in one environment may not participate in another. The total number of cells active in a given environment is proposed to be 25-30 \% providing a sparse representation which increases storage capacity and reduces interference between representations\cite{Marr2007, Wilson1993}. Recent $Ca^{2+}$ imaging of CA1 pyramidal cells are consistent with this account \cite{Ziv2013}.
Place cells are found in the CA1 and CA3 regions of the dorsal hippocampus. They have also been found in the ventral hippocampus.\\
Place cell activity in response to environmental manipulations have been intensively studied. The trajectories traversed by the rat determine the directionality of the place cell \cite{Save1998}. If the place cell is indeed coding for a specific location, then its firing must be independent of  head direction. The angular firing distribution is found to pass this test in the open field and any modulation of firing by head direction in a maze can be explained as a result of non-uniform sampling of head directions at every location \cite{Muller1994}. It is possible that in the open field the place cells achieve omni-directionality by associative binding of local views \cite{Sharp1991}. Hence, the place cells might develop omni-directionality through random exploration rather than directed exploration in a linear or radial maze.\\
The place cell spiking times are known to show reliable relationship with respect to the ongoing theta rhythm when the rat is in motion. The firing of the place cells show theta phase precession in one dimensional tracks \cite{O'Keefe1993} as well as in unconstrained open field \cite{Skaggs1996c} [one more ref]. When the animal enters the place field of a cell, the cell fires at late phase of theta and as the rate traverses through the place field it fires at earlier phases of successive theta cycles. The cell fires maximally at the trough of theta when the rat is at location corresponding to the peak of the place field tuning curve. Thus the theta phase of the spike times is correlated with the distance the animal has travelled along the place field. This is often referred to as temporal coding \cite{Huxter2003}. When several cells are recorded, the cells with overlapping place fields fire at different phases of theta, which is suggested as an encoding strategy for compressing sequences in time. This has the advantage that these sequences are arranged in the time scale of LTP to influence the synaptic connectivity between the cells participating in the sequence. Thus providing a mechanism for rapid encoding of current experience. \\
The place cells are not associated only with visual cues. The place fields are observed to persist even after the removal of visual landmarks \cite{Kubie1987}. In an experiment where the animals were placed with lights on and subsequently turned off, most place fields did not disappear in the dark \cite{Muller2008}. So the place fields are derived by integrating both visual and non-visual cues. \\ 

\begin{itemize}
Mehta, Barenes and Mc Naughton1997 - pfs shift backwards after repeated traversals along the same trajectory 
Place cell firing is also correlated with speed, direction, and turning angle  and stage of the task \\
Markus 1995 - pfs are task sensitive, pfs rapidly changed when the task changed to random foraging and search for food at the corners of a diamond\\

extra hippocampal cells have also been reported, but show different characteristics the hippocampal place cells\\
Wilson MccNaughton 1993 - place cells recorded with tetrode shower fewer subfields\\

\item Huxter, Csicsvari 2008, Nat.Neur; Theta phase-specific
codes for two-dimensional position, trajectory and head-
ing in the hippocampus.

\end{itemize}
 phase precession
 
\subsection{Head Direction cells}
Often when someone is given a map and asked to find his way to a certain goal location, one starts by orienting oneself with the help of a compass or cues such as the sun or the starts. Thus for successful spatial navigation and route planning with the cognitive map, there must be a neural system which provides a stable reference direction in an environment. A subset of cells in the postsubiculum  are found to be sensitive to head direction \cite{Taube1990}. The head direction tracking can be achieved through angular integration. The head direction system is perhaps wired to implement a continuous ring attractor which receives vestibular and visual inputs. These inputs are integrated to generate a stable activity distribution along the perceived head direction. If there is considerable mismatch between the inputs will result in the reference direction being reset along the orientation implied by the stronger input \cite{Valerio2012}.

\subsection{Grid Cells}
The neurons in the Medial Entorihnal cortex (MEC) are also found to be spatially tuned. These cells fire at regular distances along the animals trajectory. Their tuning curves show gird like structure spanning the whole environment. The vertices of the grid form equilateral triangles \cite{Hafting2005}. The grid cell population comprises grid cells of different scales, orientations and phases. They provide an internally  calibrated scale for spatial computations. How the grid structure emerges is yet to be understood. The hippocampal back projections to MEC are critical for the reliable grid structure \cite{Bonnevie2013}. 

\subsection{Path integration}
\label{pathIntgr}
Path integration, also referred to as dead reckoning is one of the strategies that can be used for spatial navigation. It keeps track of the the current location with respect to a reference point in the current reference frame. The ability to path integrate is vital for the survival of animals, it provides a means of computing the home location after the animal has traveled away from home location in search of food.
Self motion information (speed and acceleration) and heading direction are essential information that need to be available for path integration. If the speed and heading direction is known at an arbitrary location, then the future location can be readily computed. Unless the ideothetic information is noisy, a perfect path integration system is capable of maintaining location information with respect to a reference point. Any noise that enters the measurements of speed and direction will lead to accumulative error in the predicted location. This increasing error can be corrected if the system has access to location information with respect to external landmarks. In animals this is the sensory input that is constantly available. Both self motion cues and sensory information when used together provide a better estimation of the current location. Interesting experiments can be designed in virtual reality by introducing appreciable conflict between sensory input and ideothetic information. 
%Interesting phenomena occur in the place field characteristics when there is appreciable conflict between sensory input and the path integration estimation.

\subsubsection{Multichart architechture}
\label{pathIntegration}
One of theories proposes that the synaptic connections of the CA3 pyramidal cells are pre-configured to represent a large number of two-dimensional surfaces \cite{Samsonovich1997}. The pre-configured architecture is achieved through special synaptic connectivity. Along each surface the synaptic strength between neurons any two neurons is a function of the distance between the peaks of place fields.  %These two dimensional manifolds in the neural space form continuous attractors with localized activity patterns as their stable states. 
A simple way to understand how the weights are assigned is to imagine a plane with randomly chosen subset of cells placed on this plane. Then the synaptic strength between any proximally located pair of cells is given by a two dimensional gaussian function of the distance between them. The cells further away receive inhibitory connections. This synaptic connectivity scheme produces 2 dimensional quasi-continuous attractors. The imaginary arrangement of a population of place cells on an abstract plane such that each  cell is fixed at a location where it shows maximum firing activity is called a \emph{chart}. When this chart is bound to an environment, then each cell shows peak firing at the appropriate location in the physical location to which it has been mapped. Their model proposes that multiple \empth{charts} are available which could be potentially used to represent several environments or the same environment in differing contexts. These \emph{charts} exhibit no significant correlations, thus reducing  interference between representations. A \empth{chart} is referred to as the \emph{active chart} if the activity of cells in that chart appears to be localized at a specific location on the that chart. The active neurons that proximally in the active chart would be allocated random positions that are different on different charts. So the activity distribution on all other charts would appear to be dispersed. The distribution of activity can remain localized even in the absence if external stimuli since it is a stable state of the network. The active chart would then be the current reference frame used for spatial processing. Other charts would would display rand activity as dictated by the synaptic connections. \\
In their model place cells connect to a PI(path integration system) which sends back asymmetric projections to cells along the direction of motion of the animal. Thus the localized activity pattern in the active chart follows the actual movement of the animal.

%----------------------------------------------------------------------
% subsection REMAPPING
%----------------------------------------------------------------------
\subsection{remapping}
\label{remapping}
When certain features of the environment are varied by small amounts the place cell firing characteristics are altered \cite{Kubie1987}. The firing rates are observed to show drastic changes. Some place cells change their place fields to new locations while others completely vanish and new fields emerge. This is often referred to as \emph{remapping}, resulting from the changes in the spatial information of the environment. Thus distinct representations are formed for differing environments and even similar environment with apparently minor modifications. \\
Storage of memory requires that the each unique memory engram produces decorrelated patterns of activity. If the memories share some common features, it will lead activation of very similar activity patterns and the distinction between them might be lost. This is often referred to as interference and is a possible mechanism of forgetting. To keep memories distinct, a decorrelation operation of the overlapping memories must be performed to so as to produce orthogonal representations before storage. Remapping might be a signature of the underlying decorrelation computations occurring while memories are encoded.\\
Two types of remapping can be identified. Rate remapping occurs when the location of the place fields remain essential the same while the firing rates change. Global remapping occurs when the cells arbitrarily change their firing rates and develop new place fields. These categories of remapping signal different kinds of environmental manipulations. Rate remapping was observed when the location was unchanging while the color and shape of enclosures were changed. Global remapping was induced in identical enclosures at different locations \cite{Leutgeb2005a}. Changing cue configurations also produce global remapping \cite{Leutgeb2005a}.
Rate remapping probably occurs when only the relevant non-spatial features change while the spatial structure is preserved. Thus the spatial information content in the population is identical for rate remapped representations since it is only the non-spatial component of the stimuli is varying. In the $Ca^{2+}$ imaging study \cite{Ziv2013}, $15-25 \% $ of CA1 cells were found to consistently encoding for space over weeks. Rate remapping could then also be the short time scale characteristic of the ensemble encoding scheme where the population of cells recruited are changing. This would enable the encoding of other non-spatial information as episodes in an unchanging spatial setting. The only overlapping patterns in the ensemble code would be coding for the unchanging spatial information. \\
Global remapping is almost always induced when the location is changed. The magnitude of dissimilarity between the relevant environmental features dictates the likelihood of global and rate remapping. \\
The earlier experiments made discrete changes in the environment. The naturally occurring stimuli are usually continuous. The hippocampal network then has to generate dynamically varying patterns to encode the continuously changing stimuli. In a study by Leutgeb \cite{Leutgeb2005}, flexible enclosures were gradually morphed from a square shape to circle with several intermediate stages. The animals were initially familiarized with both square and circular enclosures so as to allow for stable representations of each to be learnt. Rate remapping was observed for the gradual changes from square to circle.  [transitions in the representation] [teleportation]\\
In another study \cite{Wills2005}, the square and circular enclosures were designed to induce global remapping. There was abrupt transition to a learnt representations based on the similarity of the current environment to the previously acquired one. \\
The CA3 pyramidal cells are observed to produce significantly distinct patterns in response to apparently minor changes in the environment. Since the CA3 network is also thought to function as an auto-associative memory, this would be an advantageous feature increasing the network capacity. The encoding scheme has to produce non-redundant codes for effective performance of the auto-associative network. The Dentage Gyrus (DG) input seems to be critical for pattern separation in the CA3. The DG granulae cells are hypothesized to perform the computational operation of orthogonalization of the input patterns before feeding them to the CA3 network. The sparse yet effective connectivity between the granulae cells and the CA3 pyramidal cells could provide the critical architecture.\\ The grid cells in the Medial entorhinal cortex(MEC) show little change in the grid parameters whenever the environmental change produces rate remapping in CA3. But when global remapping occurs in CA3, the grid cells show coherent shift and rotation of their fields. The time course of grid realignment and remapping seem to be closely follow each other \cite{Fyhn2007}. Perhaps the grid realignment in the MEC contributes to global remapping in the CA3. Since the CA3 also back projects to the MEC, it is also not known whether the realignment of the grids occurs first in the MEC and then leads to CA3 remapping or vice versa. Remapping the is less pronounced in CA1 in comparison to CA3 \cite{Leutgeb2005a, Leutgeb2004}. It also possible that the CA1 representation changes over a longer time scale. \\ 
The interneuron network assists the formation of new cell assemblies and supression of old ones \cite{Dupret2013}, playing an important role in the emergencce of remapped representations.

%---------------------------------------------------------------------
% subsection REPLAY
%---------------------------------------------------------------------
\subsection{Replay}
\label{replay}
The hippocampus is thought to be a temporary storage for memory \cite{Buzsaki1989, Battaglia2011}. Then the hippocampus has to meet the computational demands of rapid encoding and later recapitulation of salient memory traces for transfer to long term storage. The phenomenon of replay is a possible mechanism through which consolidation might be achieved.  Marr \cite{Marr2007} proposed that long term storage and classification of information as the functional role of neocortex. He also proposed that the neocortex would be required to be trained during sleep to classify overlapping pieces of information. Since the hippocampus has suitable anatomical communication channels via the entorhinal cortex, it is appropriate for short term storage. The Entorhinal cortex in turn has reciprocal connection with other regions. \\

Replay is the phenomena where the cell sequences that are activated along a trajectory are later activated in the same or reverse order when the animal is sleeping or even when the animal is awake during periods of immobility . The synaptic connectivity of the cells participating in a sequence during exploration are strengthened.  The sequences appearing when the animal is exploring the environment have greater propensity to occur during subsequent sleep. Significant correlated temporal bias favouring the same cell sequences to repeat during SWRs in the SWS after exploration on a linear track has been reported \cite{Skaggs1996b}. \\






%% Chapter Template

\chapter{Dynamical systems framework} % Main chapter title

\label{dynamics} % Change X to a consecutive number; for referencing this chapter elsewhere, use \ref{ChapterX}

\lhead{Chapter 2. \emph{Dynamical systems}} % Change X to a consecutive number; this is for the header on each page - perhaps a shortened title

%----------------------------------------------------------------------------------------
%	SECTION 1
%----------------------------------------------------------------------------------------

This chapter will review some concepts from dynamical systems and how these are being used to understand hippocampal population activity. Dynamical systems theory deals with time varying systems. It dwells on questions of how the behavior of systems evolve over time. Non-linear dynamics is a powerful analytic tool for studying complex systems. A network of neurons can be formalized as a system of differential equations, lending them to be studied in the same framework. There is the long standing idea that interesting abstract properties emerge in a population of interacting objects obeying certain local rules. This framework might prove to be useful in developing further insights into the mechanisms that give rise to emergent cognitive faculties.\\
The electrophysiology of the hippocampus has revealed several internally self generated patterns. Some of which are thought to be essential for memory consolidation. Although this is not restricted only to the hippocampus. The brain in general exhibits internally self generated activity even in the absence of external stimulus. The usual examples include the recordings form humans and animals during sleep. Although during sleep the sensory input is at a minimum, these recordings show a lot of interesting phenomena.
 
\section{Basic ideas}

\subsection{Differential equations}
Differential equations provide a convenient and straight forward technique to formalize and study dynamical systems. 
The variables in these are called state variables. Solving these equations given initial conditions we can predict the future states of the system.
[...................]
Higher order differential equations can be converted to a system if first order differential equations. 
The system of differential quations can be imagined as a vector field in the state space where a vector is assigned to every point in the state space. The vector points in the direction of change of the state variables. 

\subsection{Fixed points}
Fixed points are the states where there is no change or where the flow is zero \st{this figure shows a one dimensional system whose vector field is defined by some arbitrary function $f(x)$} When $f(x)$ is possitive the system moves towards $+ \inf$ when $f(x) = 0$, there is no change. These states are called the fixed points of the system.\st{ A fixed point is stable if a small perturbation causes the system to be pulled back to the fixed point...... An unstable fixed point  A mechanical equvivalent is atn inverted pendulum}

fig- show a two dimensional system where different types of fixed points result when the parameter $a$ is varied. These are the stable nodes star node , line of fixed points , saddle node. \\

A usual procedure used to analyze and visualize dynamical process is to compute the energy function for the system, if it exists. The standard picture is to imagine a ball rolling down the hill towards the valley. The energy landscapes can be complex in biological systems. Some systems are highly sensitive to initial conditions. fig - the ball can go either way depending on where it starts. \\
The stable fixed points of the systems are interesting since it they exist, The transients die out and the system will eventually settle to one of the stable states. These sable states are called attractors of the system. The topology of the attracctors can be very interesting and helpful in predicting the behaiour of the system in different states. \\
point. line attractor If end points of the line attractor are joined, a ring attractor is obtained which is proposed a model of the head direction system. \\
limit cycle attractor gives rise to periodic behaviour of the system, Limit cycles cannot occur in linear systems. Biological pattern generators can be understood in terms of limit cycles. Here is an exaple of modified HH model... x axis is tteh membrane voltage , y is teh potassium activation variable, Initially th model neuron is at rest correspondingf to a stable point. If a stronf pulse of current is injected in the membrane , it will take the neuron to the basin of attraction of the limit cycle and the neuron will produce rhythmic spikes. \\


\subsection{Stability analysis}
The stability of fixed points can be analyzed by linearizing the system at the fixed point and tehen applying linear stability analysis techniques. This gives correct predictions of stability .....
It can be shown that the ....hyperbolic fixed point....the stability is is correctely predicted by linearization.....\\when linear fails Layapnov stability analysis approach is usually adopted. Lyapnov funcion provides a generalized energy landscape and conservative estimate of domains of attraction . Here is an example of bistable system which has been used for modelling working memory. one of the states is the resting state, other is
 with persistant activity which could function as working memory.
\\
\subsection{Bifurcations}
Bifurcations occur when the system changes its qualitative behavior . Like when a solid changes to liquid state. Bifurcations reflect the dependence on parameteres. When a parameter is varied at a critical value the dynamics of the system might drasticlly change. In neural networks, the same network could beswitched betweendifferent regimes to implement different computations. Bifurcations are classified as local and global. Global if the bifurcation effects a large portion of the state space. 

Saddle node bifurcation occurs when a stable and an unstaable node collide and disappear. ...fig as the parameters r is varied the fixed points get closer and closer, then collide and the fixed points vanish. This can be dipicted as a bifurcation diagram. The parameter is teh independent variable and the fixed points are plotted as the dependent variable. The dotted lines indicate the unstable nodes and the stable mode.

pitchfork bifurcation occurs in symmetric systems as the parameter r is varied it loses stability and two new fixed points are created, The bifurcation diagrams are show fig.... eg - a beam with load

Hopf bifurcation occurs when a stable spiral loses its stability and a limit cycle is created. .fig--- model neuron the injected current as the aprameter , when ramu current is applied at a critical value the spiking........

\section{Attractor networks}
attractor networks have stable patterns as their attractors and depending on the initial conditions the network will settle down to one of the stable patterns. Depending on the type if attractor different kinds patterns can be achieved. It has been proposed the recurrent network in the CA3 might function as an auto-associative network........... The characteristics of the attractor determines which patterns are stable.
This type of continuous attractor dynamics has been proposed as a mechanism for path integration (sec. \ref{pathIntgr}, p. \pageref{pathIntgr}). Localized activity patterns are achieved by local excitation and long range inhibition. In networks  with shift invariant structures it is possible to stabilize activity pattern on each node. Thus if we have a line of attractors in the limit of infinite nodes a continuous manifold of point attractors can be generated.

\subsection{Auto-associative network}
fig - auto associative network which has learnt a few patterns and when presented woth noisy patterns... Each pattern can be thought of as a minima in the energy landscape. Pattern completion occurs because of the recurrent synaptic architecture. When a few neurons of a certain pattern are active, then they activate others because of strong synaptic connectivity strength. The 


\subsection{Models of place selectivity and phase precession}
The basic requirement to get place cell is localized firing patterns, so that the this pattern might then be utilized to encode distinct locations in the physical space. Second requirement is to have a mechanism by which this localized activity can be conveniently updated as the animal traverses in the environment.\\

Contrary to the view that there is an explicit temporal code (sec. \ref{placeCells} p. \pageref{placeCells}), it has been suggested that the correlation between position and phase of theta could be explained as a consequence of sequential computation occurring within a theta cycle. Across the population of place cells, different phases the theta cycle encode positions offset into either future or past along the rat's trajectory \cite{Itskov2008}. The past and future locations  robustly predict the CA1 place cell activity at different phases. This indicates that the information content in the cell activity at different phases of theta is actually correlated with the past and future locations of the animal. It was also show that this phenomena is not a direct consequence of phase precession, rather might actually be a causing the observed phase precession. \\

Tsodyks \cite{Tsodyks1996} proposed a model proposing possible mechanisms through with place cell selectivity is achieved through a specific synaptic connectivity between neurons reflecting the distances between their place field peaks in the environment. In the presence of global inhibition this architecture results to attractor dynamics. External input with weak selectivity is sufficient to steer the network into the one of the basins of attraction resulting in stable localized activity. The activity tracks the input peak location as it moves. With the addition of asymmetric synaptic strengths in the direction of motion phase precession effect is observed. Hence the phase precession could be a manifestation of the inherent asymmetry in the synaptic connections. The inhibitory interneurons in the model network receive theta input which is the simulated input from medial septum to the GABAergic interneurons. The simulation of the network shows that as the rat moves along the track the external excitation drifts through a group of neurons. Due to the asymmetry in the synaptic connections the activity spontaneously propagates forward in every theta cycle. \\  

\subsection{Internal sequences in the rat hippocampus}
In the study by Pastalkova \cite{Pastalkova2008a}, multi-unit recordings were obtained form the rat hippocampus in non-sleep state. They report observation of internally self sustained cell assembly sequences. The animals were trained to run in a wheel during the delay period in an alternation task. The CA1 pyramidal neurons were recorded during the delay period. Some of these neuron assemblies were sequentially activated. These sequences were predictive of the time the rat spent in the wheel upto 20 seconds \cite{Itskov2011a}. The sequences were unique for different behavioral choices including the ones which were incorrect choices. Since the location of the rat was stationary, one would expect to see only place cells for that location to be active.  \\  
[time/distance cells]
It has been proposed that this could be a means by which the networks keeps track of time elapsed. A model was proposed suggesting a possible mechanism for the generation of cell sequences in a network with no strong inputs.  The two critical ingredients of the model are adaptive thresholds and Mexican hat synaptic connectivity. In the model, every spike fired by the neuron will result in the increase of its spiking threshold. This threshold then exponentially decays to it default value with a time constant in the order of seconds. Thus a neuron gets increasingly discouraged to fire as its firing rate increases, which will eventually silence the neuron.The Mexican hat type connectivity ensures that the activity remains localize to a small number of neighboring neurons. The symmetry in the connectivity is broken by introducing uncorrelated noise in the connectivity matrix. In fact, to ensure that the dynamics displayed by the model is not just a result of the perfect synaptic tuning; the strength of the correlated Mexican hat connectivity was chosen to be weaker as compared to that of the heterogeneous component. This model generates a continuum of bump attractors, for the network, this implies that if the bump were to be moved laterally by some means, it would stabilize in the new position.  The introduction of threshold adaptation will result in the increase of the thresholds of neurons participating in the localized activity. Gradually, on the time scale of seconds (i.e. the time scale of threshold relaxation), this bump will lose its stability as the neurons become quite due high values of threshold. Assuming the input noise levels are low, the heterogeneity in the network connectivity will dictate the next position of the stable bump, which will remain so until it is destabilized again by the same mechanism. The bump in essence is moving away from the neurons with recently updated thresholds. In this setting, activity bump shifts its peak constantly without ever stabilizing at a particular location. Thus the model exhibits self generated sequential activation of cells, captured by the bump constantly moving along continuous trajectories in the state space. The model produces reliable trajectories even with weakly noisy input, provided that it starts with the same initial conditions. Hence, the similar heterogeneity and threshold levels across trials provide identical contexts which results in reproducible behavior. It was also shown that these sequences can be inherited by succeeding layer of neurons without recurrent connections receiving sparse feed-forward input from the layer with recurrent connectivity.Since the CA1 region does not have recurrent connectivity and the sequences were observed in CA1, it is possible that they are inherited form sequences are generated elsewhere.


Further as 
The idea of continuous attractors are used to model these cell sequences.   ....



%\section{Objectives}
[INCOMPLETE]\st{
This study was aimed at characterizing the spatio-temporal dynamics of the hippocampal network. The place cell activity dynamics across multiple environments were investigated. How do the firing 


Since 2 dimensional environments are closer to the spatial layout in the natural world, the place cells in 2D arena was analyzed in this study. The population activity in the CA3 and CA1 were analysed 


 The place cells in 2D arena was analyzed in this study.}\st{find evidence for the existence of attractors in the hippocampal network. 
search for the computational algorithms used in the hippocampus and its supporting structures for path integration and episodic memory. how well does the data agree with the existing models of path integration and remapping}


