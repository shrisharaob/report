\section{Objectives}
The purpose of this study was chiefly to perform exploratory data analysis leading to the characterization of the spatio-temporal dynamics of the CA3 and CA1 hippocampal networks. This study was aimed at answering some of the  open questions related to how the principal cell population in the hippocampus alter their firing properties and consequently their place field features. One of the goals was to investigate how the change of population activity biases the activity during sleep and visa versa, which is an important question directly related to how the internal dynamics and the stimulus related dynamics interact. 
\begin{itemize}
\item \emph{The place cell activity changes across multiple environments and the corresponding place field characteristics.}
Whenever a cell is participating in the spatial representation of two or more distinct environments, how is its activity across environments related. It Are there any features of the place fields that are invariant  different environments.
\item \emph{Changes in firing properties, if any; that occur in the same environment across multiple sessions.}
In a given and unchanging environment, it is interesting to ask weather the firing characteristics of single cells are reliable across multiple sessions and even days.
\item \emph{Pairwise firing characteristics of place cells in relation to spatial outlay of their receptive fields.}
Once stable place fields have been established, the sequential activation of cells is observed for the place cells that have overlapping spatial receptive fields. When pairs of cells with overlapping place fields are considered, these pairs would show high short time correlation in their firing.
\item \emph{The dynamics of population of place cells that are active within and across different environments.}
If each familiar environment is represented in the synaptic matrix of the CA3 as a 2D continuous attractor (p. \pageref{fixedpt}), does the time course of the population activity when that animal is exploring have any special characteristics i.e how the population activity evolves in time. Does the population activity of a subset of cells participating in several distinct environment show any commonalities.

\item \emph{Cells sequences during rest.} 
Cell sequences in 1 and 2 dimensional environments that occur during sleep and immobility and their relation to sequences generated during cell sequences occurring when the animal is exploring the environment. 
\end{itemize}
 