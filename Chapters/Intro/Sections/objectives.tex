\section{Objectives}
[INCOMPLETE]
This study was aimed at characterizing the spatio-temporal dynamics of the hippocampal network.
\begin{itemize}
\item The place cell activity changes across multiple environments. 
\item Changes in firing properties, if any; that occur in the same environment across multiple sessions.
\item Features of the place cell activity that are preserved in different environments.
\item Pairwise firing characteristics of place cells in relation to spatial outlay of their receptive fields.
\item Cell sequences in 1 and 2 dimensional arenas that occur during sleep and immobility and their relation to sequences generated during cell sequences occurring when the animal  is traversing. 
\end{itemize}
 
\st{

Since 2 dimensional environments are closer to the spatial layout in the natural world, the place cells in 2D arena was analyzed in this study. The population activity in the CA3 and CA1 were analysed 


 The place cells in 2D arena was analyzed in this study.}\st{find evidence for the existence of attractors in the hippocampal network. 
search for the computational algorithms used in the hippocampus and its supporting structures for path integration and episodic memory. how well does the data agree with the existing models of path integration and remapping}
