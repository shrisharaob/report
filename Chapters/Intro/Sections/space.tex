\section{Internal representation of space} 

\label{space} 

\lhead{Chapter 1. \emph{Internal Representation of space}} % This is for the header on each page

%----------------------------------------------------------------------------------------
Spatial representation in animals is vital for their survival since they have to search for food and mates, this makes considerable demands in terms of spatial processing for success. Representation of space requires that the geometric relationship between objects in the environment are encoded using a convenient metric. The relevant objects and landmarks in the environment can thus be organized in a spatial framework, although the structure of space is independent of the sensory input through which the former is extracted. Spatial information provides the context for adaptive behaviours. Storing spatial relationships over time could be a plausible framework for encoding episodic memory. \\
Egocentric or allocentric representations can be used to encode and store spatial information. Egocentric space corresponds to the representation relative to the viewer's current location. If the viewer moves, the egocentric representation also changes correspondingly. In allocentric representation, the spatial relationships of landmarks and objects in the environment are encoded relative to each other, independent of the viewer. Unless the objects move, the allocentric representation is stable even when the viewer moves.\\
Regardless of the representation employed, the puzzling question of how this representation forms still remains unanswered. Philosophers have long pondered over the origins of spatial representation. There exist theories that postulate an internal pre-existing spatial map is bound to the external stimuli in a given environment. This is along the lines of Kantian metaphysical exposition on space, where he argues that space is not an empirical concept derived from sensory experience, rather that it already exists \emph{a priori}. Other theories postulate that the simple associations between incoming multi sensory stimuli result in the formation of spatial maps. These theories are along the classical empiricist idea of space. 

\subsection{Spatial navigation in the the rat}
In this section I shall review some of the prominent discoveries made in experiments of spatial navigation conducted on rats and mice. Several strategies can be adopted to perform spatial navigation. It is thought to be achieved primarily by path integration mechanism. The place cells, head direction cells and the grid cells might provide the necessary ingredients to achieve path integration. 

\subsubsection{Place cells}
\label{placeCells}
 \emph{Place cells} were first discovered by O'Keefe and Dostrovsky \cite{O'Keefe1971a}. Following this discovery, O'Keefe and Nadel \cite{Street} conceived of a functional role for the hippocampus, they proposed that the hippocampus is the substrate for a cognitive map which is formed while the rat explores its environment (\emph{cognitive map theory}). They suggested that the primary function of the hippocampus was to encompass mechanisms that would enable the learning and storage of the map in the hippocampal network. The cognitive map represents spatial information in allocentric space. So the relative spatial metrics of the landmarks in the incoming stimuli have to be computed. This map can then be utilized for solving navigational tasks while the various structures of the hippocampus provide the spatial information related to place, head direction, and current view. They also suggested that the map is stored in the hippocampus and it is used in conjunction with information stored in other areas. By extending their theory to humans they postulated that episodic memory derives its basis from similar mechanisms that enable spatial processing in animals.\\ 
Place cells are the  pyramidal cells in the rat and mice hippocampus. Place cells derive their name form the fact that their discharge is strongly correlated with the position of the animal in the environment. In a given unchanging environment, each cell has a preferred location called the place field where the cell fires maximally.  The cells that are active in one environment may not participate in another. The total number of cells active in a given environment is proposed to be 25-30 \% providing a sparse representation which increases storage capacity and reduces interference between representations\cite{Marr2007, Wilson1993a}. Recent $Ca^{2+}$ imaging of CA1 pyramidal cells is consistent with this account \cite{Ziv2013}.
Place cells are found in the CA1 and CA3 regions of the dorsal hippocampus. They have also been found in the ventral hippocampus. \\ 
Place cell activity in response to environmental manipulations have been intensively studied. The trajectories traversed by the rat determine the directionality of the place cell \cite{Save1998}. If the place cell is indeed coding for a specific location, then its firing must be independent of  head direction. The angular firing distribution is found to pass this test in the open field and any modulation of firing by head direction in a maze can be explained as a result of non-uniform sampling of head directions at every location \cite{Muller1994}. It is possible that in the open field the place cells achieve omni-directionality by associative binding of local views \cite{Sharp1991}. Hence, the place cells might develop omni-directionality through random exploration rather than directed exploration in a linear or radial maze.\\
The place cell spiking times are known to show reliable relationship with respect to the ongoing theta rhythm when the rat is in motion. The firing of the place cells show theta phase precession in one dimensional tracks \cite{O'Keefe1993} as well as in unconstrained open field \cite{Skaggs1996c}. When the animal enters the place field of a cell, the cell fires at late phase of theta and as the rate traverses through the place field it fires at earlier phases of successive theta cycles. The cell fires maximally at the trough of theta when the rat is at location corresponding to the peak of the place field tuning curve. Thus the theta phase of the spike times is correlated with the distance the animal has traveled along the place field. This is often referred to as temporal coding \cite{Huxter2003}. When several cells are recorded, the cells with overlapping place fields fire at different phases of theta, which is suggested as an encoding strategy for compressing sequences in time. This has the advantage that these sequences are arranged in the time scale of LTP to influence the synaptic connectivity between the cells participating in the sequence. Thus providing a mechanism for rapid storage of encoded ongoing experience. \\
The place cells are not associated only with visual cues. The place fields are observed to persist even after the removal of visual landmarks \cite{Kubie1987}. In an experiment where the animals were placed with lights on and subsequently turned off, most place fields did not disappear in the dark \cite{Muller2008}. So the place fields are derived by integrating both visual and non-visual cues. Place fields are observed even during the first time exposure of the animal to a new environment \cite{Reccivcd1978, Tanila1997}. Although they require time to achieve relaible and stable representation \cite{Wilson1993a}. Thus it is not clear if the place fields emerge  as a result of binding of local cues or if they are already part of pre existing spatial maps.  
 
\subsubsection{Head Direction cells}
Often when someone is given a map and asked to find his way to a certain goal location, one starts by orienting oneself with the help of a compass or cues such as the sun or the starts. Thus for successful spatial navigation and route planning with the cognitive map, there must be a neural system which provides a stable reference direction in an environment. A subset of cells in the postsubiculum  are found to be sensitive to head direction \cite{Taube1990}. The head direction tracking can be achieved through angular integration. The head direction system is perhaps wired to implement a continuous ring attractor which receives vestibular and visual inputs. These inputs are integrated to generate a stable activity distribution along the perceived head direction. If there is considerable mismatch between the inputs will result in the reference direction being reset along the orientation implied by the stronger input \cite{Valerio2012}. 

\subsubsection{Grid Cells}
The neurons in the Medial Entorihnal cortex (MEC) are also found to be spatially tuned. These cells fire at regular distances along the animals' trajectory. Their tuning curves show gird like structure spanning the whole environment. The vertices of the grid form equilateral triangles \cite{Hafting2005}. The grid cell population comprises grid cells of different scales, orientations and phases. They provide an internally  calibrated scale for spatial computations. How the grid structure emerges is yet to be understood. The hippocampal back projections to MEC are critical for a reliable grid structure \cite{Bonnevie2013}. The grid cells might be a sign of an internal organizing framework which is independent of sensory input and exists \emph{a priori}. The external stimuli is made comprehensible by imposing spatial structure onto it \cite{Kant2003}. 

\subsection{Path integration}
\label{pathIntgr}
Path integration, also referred to as dead reckoning is one of the strategies that can be used for spatial navigation. It keeps track of the the current location with respect to a reference point in the current reference frame. The ability to path integrate is vital for the survival of animals, it provides a means of computing the home location after the animal has traveled away from home location in search of food.
Self motion information (speed and acceleration) and heading direction are essential information that need to be available for path integration. If the speed and heading direction is known at an arbitrary location, then the future location can be readily computed. Unless the ideothetic information is noisy, a perfect path integration system is capable of maintaining location information with respect to a reference point. Any noise that enters the measurements of speed and direction will lead to accumulative error in the predicted location. This increasing error can be corrected if the system has access to location information with respect to external landmarks. In animals this is the sensory input that is constantly available. Both self motion cues and sensory information when used together provide a better estimation of the current location. Interesting experiments can be designed in virtual reality by introducing appreciable conflict between sensory input and ideothetic information. These experiments among other things provide a method to study the degree of mismatch required to induce changes in the internal activity. (supposing that internal activity is resistant to small perturbations, for example in a network with attractor dynamics [p. \pageref{fixedpt}]). 
%Interesting phenomena occur in the place field characteristics when there is appreciable conflict between sensory input and the path integration estimation.

\subsection{Multiple maps and Multichart architechture}
\label{multichart}
When the path integrator estimate of current position and the the local view disagree significantly, the path integrator has to be updated to correct for the discrepancy. This correction will lead to a completely new place code for the environment, which is the same as resetting the coordinates of the path integrator. The path integrator coordinates rely on a stable reference point or direction which could be understood as providing a reference frame. The location is then coded within this reference frame or map. Thus distinct environments or the same environment in diverse contexts can each be associated with a unique reference frame. It has been proposed that multiple cognitive maps are present to accommodate for self localization and path integration in a multitude of tasks with diverse features \cite{Street}. Only subsets of place cells are active in these maps \cite{Muller1987}. The mechanisms of how the switching occurs between maps is an open issue, some recent experiments show that the CA3 cell assemblies display bi-stability between representations within individual theta cycles\cite{Jezek2011} and eventually converge to the appropriate representation in a few theta cycles.\\
 
In speculating the mechanisms for existence of multiple map, one of theories proposes that the synaptic connections of the CA3 pyramidal cells are pre-configured to represent a large number of two-dimensional surfaces \cite{Samsonovich1997}. These two dimensional surfaces represent the multiple maps. The pre-configured architecture is achieved through special synaptic connectivity. Along each surface the synaptic strength between neurons any two neurons is a function of the distance between the peaks of place fields.  %These two dimensional manifolds in the neural space form continuous attractors with localized activity patterns as their stable states. 
A simple way to understand how the weights are assigned is to imagine a plane with randomly chosen subset of cells placed on this plane. Then the synaptic strength between any proximally located pair of cells is given by a two dimensional Gaussian function of the distance between them. The cells further away receive inhibitory connections. This synaptic connectivity scheme produces 2 dimensional quasi-continuous attractors. The imaginary arrangement of a population of place cells on an abstract plane such that each  cell is fixed at a location where it shows maximum firing activity is called a \emph{chart}. When this chart is bound to an environment, then each cell shows peak firing at the appropriate location in the physical location to which it has been mapped. This model proposes that multiple \emph{charts} are available which could be potentially used to represent several environments or the same environment in differing contexts. These \emph{charts} exhibit no significant correlations, thus reducing  interference between representations. A \emph{chart} is referred to as the \emph{active chart} if the activity of cells in that chart appears to be localized at a specific location on the that chart. The active neurons that are proximally located in the active chart would be allocated random positions that are different on different charts. So the activity distribution on all other charts would appear to be dispersed. The distribution of activity can remain localized even in the absence of external stimuli since it is a stable state of the network. The active chart would then be the current reference frame used for spatial processing. Other charts would would display rand activity as dictated by the synaptic connections. \\
In this model place cells connect to a PI (path integration system), which sends back asymmetric projections to cells along the direction of motion of the animal. Thus, the localized activity pattern in the active chart follows the actual movement of the animal. Importantly, this model postulates that the charts are pre-wired in the hippocampus and the sensory cues in each environment is associated with these maps (sec. \ref{autoasso}).\\
The following section on remapping could be understood in terms of the representation switching between multiple reference frames. 
%----------------------------------------------------------------------
% subsection REMAPPING
%----------------------------------------------------------------------
\subsection{Remapping}
\label{remapping}
The presence of multiple reference frames with a certain subset of cells in the population active in each frame could be tested by manipulating the external environmental cues expecting noticeable changes in the activity of different subsets of cells in multiple distinct cue configurations and environments.\\

When certain features of the environment are varied by small amounts the place cell firing characteristics are altered \cite{Kubie1987}. The firing rates are observed to show drastic changes. Some place cells change their place fields to new locations, while others completely vanish and new fields emerge. This is often referred to as \emph{remapping}, resulting from the changes in the spatial information of the environment. Hence, distinct representations are formed for differing environments and even in a similar environment with apparently minor modifications. \\
Storage of memory requires that each unique memory trace is encoded as decorrelated patterns of activity in the network. If the memories share some common features, it will lead to activation of very similar patterns and the distinction between them might be lost. This is often referred to as interference and is a possible mechanism of forgetting. To keep memories distinct, a decorrelation operation of the overlapping memories must be performed so as to produce orthogonal representations before storage. Remapping might be a signature of the underlying decorrelation computations occurring while memories are encoded.\\
Two types of remapping can be identified. When the location of the place fields remain essential the same while the firing rates change, it is referred to as \emph{rate remapping}. When the cells arbitrarily change their firing rates and develop new place fields, \emph{Global remapping} is said to have occurred. These categories of remapping signal different kinds of environmental manipulations. Rate remapping was observed when the location was unchanging while the color and shape of enclosures were changed. Global remapping was induced in identical enclosures at different locations \cite{Leutgeb2005a}. Changing cue configurations also produce global remapping \cite{Leutgeb2005a}.
Rate remapping probably occurs only when the relevant non-spatial features change while the spatial structure is preserved. Thus the spatial information content in the population is identical for different rate remapped representations since only the non-spatial component of the stimuli is varying. In the $Ca^{2+}$ imaging study \cite{Ziv2013}, $15-25 \% $ of CA1 cells were found to consistently encoding for space over weeks. Rate remapping could then also be the short time scale characteristic of the ensemble encoding scheme where the population of cells recruited are changing. This would enable the encoding of other non-spatial information as episodes in an unchanging spatial setting. The only overlapping patterns in the ensemble code would be coding for the temporally invariant spatial information. \\
Global remapping is almost always induced when the location is changed. The magnitude of dissimilarity between the relevant environmental features dictates the likelihood of global and rate remapping. \\
The first experiments usually made discrete changes in the environment. The naturally occurring stimuli are usually continuous. The hippocampal network then has to generate dynamically varying patterns to encode the continuously changing stimuli. In a study by Leutgeb \cite{Leutgeb2005}, flexible enclosures were gradually morphed from a square shape to circle with several intermediate stages. The animals were initially familiarized with both square and circular enclosures so as to allow for stable representations of each to be learnt. Rate remapping was observed for the gradual changes from square to circle.  [transitions in the representation] [teleportation]\\
In another study \cite{Wills2005}, the square and circular enclosures were designed to induce global remapping. There was abrupt transition to a learnt representations based on the similarity of the current environment to the previously acquired one. The representations also change when the behavioural task the animal has to perform changes \cite{Markus1995}. \\
The CA3 pyramidal cells are observed to produce significantly distinct patterns in response to apparently minor changes in the environment. Since the CA3 network is also thought to function as an auto-associative memory, this would be an advantageous feature increasing the network capacity. The encoding scheme has to produce non-redundant codes for effective performance of the auto-associative network. The Dentage Gyrus (DG) input seems to be critical for pattern separation in the CA3. The DG granule cells are hypothesized to perform the computational operation of orthogonalization of the input patterns before feeding them to the CA3 network. The sparse yet effective connectivity between the granule cells and the CA3 pyramidal cells could provide the critical architecture.\\ The grid cells in the Medial entorhinal cortex(MEC) show little change in the grid parameters whenever the environmental change produces rate remapping in CA3. But when global remapping occurs in CA3, the grid cells show coherent shift and rotation of their fields. The time course of grid realignment and remapping seem to be closely follow each other \cite{Fyhn2007}. Perhaps the grid realignment in the MEC contributes to global remapping in the CA3. Since the CA3 also back projects to the MEC, it is also not known whether the realignment of the grids occurs first in the MEC and then leads to CA3 remapping or vice versa. Remapping the is less pronounced in CA1 in comparison to CA3 \cite{Leutgeb2005a, Leutgeb2004}. It also possible that the CA1 representation changes over a longer time scale. The interneuron network assists the formation of new cell assemblies and supression of old ones \cite{Dupret2013}, playing an important role in the emergencce of remapped representations.\\ 

Thus remapping experiments support the multiple map hypothesis which argues that for different environments and/or contexts a unique representation is achieved through switching between a reservoir of pre-configured reference frames which get associated with the environment and/or context by experience.
%------------------------------------------------------------------
% subsection REPLAY
%---------------------------------------------------------------------
\subsection{Reactivation of activity during sleep}
\label{replay}
The recordings of hippocampus during sleep has provided some crucial support to the idea of internal self generated dynamics having functional role in organizing the computations occurring in the brain. It has been observed that some of the activity pattern that arises when the animal is awake manifests itself in the ensuing sleep periods \cite{Wilson1994}. \\
In rats and mice there is a noticeable reactivation of activity of place cells, which is often called \emph{Replay}. \emph{Replay} is a phenomenon where the cell sequences that are activated along a trajectory are later activated in the same or reverse order when the animal is sleeping or even when the animal is awake during periods of immobility. The synaptic connectivity of the cells participating in a sequence during exploration are strengthened.  The sequences appearing when the animal is exploring the environment have greater propensity to occur during subsequent sleep. Significant correlated temporal bias favouring the same cell sequences to repeat during SWRs in the SWS after exploration on a linear track has been reported \cite{Skaggs1996b}. \\
The hippocampus is thought to be a temporary storage for memory \cite{Buzsaki1989, Battaglia2011}. Then the hippocampus has to meet the computational demands of rapid encoding and later recapitulation of salient memory traces for transfer to long term storage. The phenomenon of \emph{replay} is a possible mechanism through which consolidation might be achieved.  Marr \cite{Marr2007} proposed that long term storage and classification of information as the functional role of neocortex. He also proposed that the neocortex would be required to be trained during sleep to classify overlapping pieces of information. Since the hippocampus has suitable anatomical communication channels via the entorhinal cortex, it is appropriate for short term storage. The Entorhinal cortex in turn has reciprocal connection with other regions. \\

Thus \emph{replay} provides a mechanism by which the activity patterns that arose during behaviourally salient events are reactivated during sleep and consolidated for future guidance of behaviour. \\


The experimental findings presented in the preceding sections give us insights into the interaction of internal dynamics and external stimuli. Although they do not yet completely answer the chief question on the origins of spatial representation (if they are learnt or innate ?). \\
The next section will be review of dynamical systems theory and how the concepts developed there is being used to gain further understanding of the network dynamics in the hippocampus. It provides methods to partially formalize the questions posed and conceptually explain the experimental observations. 






