% Chapter Template

\chapter{Dynamical systems framework} % Main chapter title

\label{dynamics} % Change X to a consecutive number; for referencing this chapter elsewhere, use \ref{ChapterX}

\lhead{Chapter 2. \emph{Dynamical systems}} % Change X to a consecutive number; this is for the header on each page - perhaps a shortened title

%----------------------------------------------------------------------------------------
%	SECTION 1
%----------------------------------------------------------------------------------------

This chapter will review some concepts from dynamical systems and how these are being used to understand hippocampal population activity. Dynamical systems theory deals with time varying systems. It deals with the questions of how the behavior of systems evolve over time. Non linear dynamical theory is a powerful analytical tool for studying complex systems. A network of neurons can also be formalized as a dynamical system lending  them to be studied in the same framework. There is the long standing idea that interesting abstract properties emerge in a population of interacting objects obeying certain local rules. \st{The approach will be to explain observed hippocampal phenomena as a result of underlying network phenomena}.
 
\section{Basic ideas}

\subsection{Differential equations}
Differential equations provide a convenient ans straight forward technique to formalize and study dynamical systems. The variables in these are called state variables. Solving these equations given initial conditions we can predict the future states of the system.
[...................]
Higher order differential equations can be converted to a system if first order differential equations. 
The system of differential quations can be imagined as a vector field in the state space where a vector is assigned to every point in the state space. The vector points in the direction of change of the state variables. 

\subsection{Fixed points}
Fixed points are the states where there is no change or where the flow is zero \st{this figure shows a one dimensional system whose vector field is defined by some arbitrary function $f(x)$} When $f(x)$ is possitive the system moves towards $+ \inf$ when $f(x) = 0$, there is no change. These states are called the fixed points of the system.\st{ A fixed point is stable if a small perturbation causes the system to be pulled back to the fixed point...... An unstable fixed point  A mechanical equvivalent is atn inverted pendulum}

fig- show a two dimensional system where different types of fixed points result when the parameter $a$ is varied. These are the stable nodes star node , line of fixed points , saddle node. \\

A usual procedure used to analyze and visualize dynamical process is to compute the energy function for the system, if it exists. The standard picture is to imagine a ball rolling down the hill towards the valley. The energy landscapes can be complex in biological systems. Some systems are highly sensitive to initial conditions. fig - the ball can go either way depending on where it starts. \\
The stable fixed points of the systems are interesting since it they exist, The transients die out and the system will eventually settle to one of the stable states. These sable states are called attractors of the system. The topology of the attracctors can be very interesting and helpful in predicting the behaiour of the system in different states. \\
point. line attractor If end points of the line attractor are joined, a ring attractor is obtained which is proposed a model of the head direction system. \\
limit cycle attractor gives rise to periodic behaviour of the system, Limit cycles cannot occur in linear systems. Biological pattern generators can be understood in terms of limit cycles. Here is an exaple of modified HH model... x axis is tteh membrane voltage , y is teh potassium activation variable, Initially th model neuron is at rest correspondingf to a stable point. If a stronf pulse of current is injected in the membrane , it will take the neuron to the basin of attraction of the limit cycle and the neuron will produce rhythmic spikes. \\


\subsection{Stability analysis}
The stability of fixed points can be analyzed by linearizing the system at the fixed point and tehen applying linear stability analysis techniques. This gives correct predictions of stability .....
It can be shown that the ....hyperbolic fixed point....the stability is is correctely predicted by linearization.....\\when linear fails Layapnov stability analysis approach is usually adopted. Lyapnov funcion provides a generalized energy landscape and conservative estimate of domains of attraction . Here is an example of bistable system which has been used for modelling working memory. one of the states is the resting state, other is
 with persistant activity which could function as working memory.
\\
\subsection{Bifurcations}
Bifurcations occur when the system changes its qualitative behavior . Like when a solid changes to liquid state. Bifurcations reflect the dependence on parameteres. When a parameter is varied at a critical value the dynamics of the system might drasticlly change. In neural networks, the same network could beswitched betweendifferent regimes to implement different computations. Bifurcations are classified as local and global. Global if the bifurcation effects a large portion of the state space. 

Saddle node bifurcation occurs when a stable and an unstaable node collide and disappear. ...fig as the parameters r is varied the fixed points get closer and closer, then collide and the fixed points vanish. This can be dipicted as a bifurcation diagram. The parameter is teh independent variable and the fixed points are plotted as teh dependent variable. The dotted lines indicate the unstable nodes and the stable mode.

pitchfork bifurcation occurs in symmetric systems as teh parameter r is varied it loses stability and two new fixed points are created, The bifurcation diagrams are show fig.... eg - a beam with load

Hopf bifurcation occurs when a stable spiral loses its stabiility and a limit cycle is created. .fig--- model neuron the injected current as the aprameter , when ramu current is applied at a critical valuee the spiking........

\section{Attractor networks}
attractor networks have stable patterns as their attractors and depending oon the initial coditions the network will settle down to one of the stable patterns. Depending on the type if attractor different kinds patterns can be achieved. It has been proposed tha the recurrent network in the CA3 might function as an suto-associative network...........

