\chapter{Discussion}
\lhead{Chapter 4. \emph{Discussion}} 

The network dynamics of the brain and its sub-systems is inherently non-linear, which might provide additional layers of complexity required to achieve cognition in a purely material substrate. The neural networks of the brain are endowed with the capability to generate reliable activity patterns without any external stimulus. These internally self generated patterns put constraints on how the network behaves when external stimuli are presented. This was the underlying framework for this study. The experimental studies in the hippocampus have shown several phenomena that are internally generated. The hippocampus is also the center for spatial information processing. The discovery of place cells \cite{O'Keefe1971a} lead to the subsequent \emph{Cognitive Map hypothesis} \cite{Street}. Since then, a lot of studies have been conducted to understand the mechanisms for the formation of the spatial representation in the hippocampus. One of the predominant open questions has been - whether the spatial representation is innate or if the external stimuli is an absolutely necessary prerequisite for its formation. To solve this puzzle, it is necessary to first characterize the relevant dynamics in the hippocampus. Certain aspects of spatial processing in the hippocampus was the focus of this thesis in anticipation of discovery of interesting structure in the dynamics of the hippocampal sub-networks. 

Due to the lack of statistically sufficient number of stable place fields with single place fields in the 2D arena, some of the questions posed could not be conclusively answered. Many of the place cells in the data set exhibited multiple place fields, which is not a characteristic of classical place cells. The multiple place fields observed could be result of the some spikes from several cells getting clustered into a single cluster i.e poor cluster isolation. An alternative explanation is provided by the ensemble encoding hypothesis, which predicts irregularly multiple place fields in large environments as a natural outcome of the encoding scheme used in to represent spatial locations. Some experimental studies have been actually conducted in to test this hypothesis \cite{Fenton2008}. 

It was found that the network activity on the whole remains at a level that is maintained throughout the periods of exploration, which is perhaps the biologically imposed homeostasis. At the single cell level this means that when some of the cells increase their firing rate, other cells are forced to reduce their firing rates. \\
Single cells with smaller firing rates show higher variance in their firing rates. The place cells with higher firing rates are expected to have lower variance in their firing rates, supposing that the magnitude of firing rate is an indicator of a stronger association to a salient location. Although not a lot of place cells with higher firing rates were available to test this idea.\\

The pairwise analysis was performed with the intention of quantifying invariant features of pairs of place cells expressed either in their firing activity or their spatial receptive fields. For every pair of cell with overlapping place fields, it was tested whether the the time offset in the spike time correlations are preserved across environments. From the available data, there was no such invariant relationship was found. The next question asked was if the pair wise distances for the the place fields are preserved across environments with different topology.\\
For cells with multiple place fields, each of the sub-fields were dealt with separately. It was found that the CCGs had peaks at several time lags and this was thought that this was due to the sub-fields. The CCGs for pairs of overlapping sub-fields were computed by selecting the spikes that were fired only inside the sub-fields to check if the offsets from the different sub-fields can be extracted. 

In one dimensional tracks it has been shown that the theta phase of spike times reliably encode the location of the animal within the place field. \cite{O'Keefe1993, Skaggs1996c, Huxter2003}. This has been demonstrated in the open field \cite{Huxter2008a}. 

to reveal how the relative firing characteristics of pairs of place cells with overlapping place fields in atleast one of  two or more distinct environments.  


%\st{%Since 2 dimensional environments are closer to the spatial layout in the natural world, the place cells in 2D arena was analyzed in this study. The population activity in the CA3 and CA1 were analysed 
%The place cells in 2D arena was analyzed in this study.}\st{find evidence for the existence of attractors in the hippocampal network.  search for the computational algorithms used in the hippocampus and its supporting structures for path integration and episodic memory. how well does the data agree with the existing models of path integration and remapping}

The Bayesian decoding of the animals position from the place field spikes had very low accuracy in the 2D arena. This is due to the low number of place cells that were available, it has been shown that the mean decoding error is inversely proportional to the square root of the number cell (this is valid for sufficiently large number of cells) \cite{Zhang2013}. For the same reason the analysis of population events in the 2D environment did not yield any significant results. The decoding position showed a lot of discontinuous jumps.\\

Since the Bayesian decoding required more cells, a simpler measure of pairwise co-firing likelihood was computed and compared before and after exploration. But, this measure provides a method to characterize only pairs of cells, and not more that form sequences. It is expected that the the cells with closer place fields have higher cofiring likelihood which would drop for larger distances, this was indeed the case.\\ 

For the linear tracks, the population events were correlated with the templates constructed from the the cell sequences that fired when the animal traversed the linear track. There were (numbers/stats...) events in the pre-trial sleep that had significant correlation with the template and there were () events that correlated with the post-trial sleep. These significant events were detected at a predetermined threshold, it might lead to the false negatives for events with lower correlation values. So the difference between the cdfs were used and the significance of the differences were computed. Subsequently, the frequency of occurrence of the significant events was examined, but a small number of such events prevented further examination.\\
It was observed that the majority of cells that fired during population events were not active when the animal was running on the linear track. To test if these cells have reliable firing sequences, the sequences were correlated with each other and sampling distributions for these correlations were computed by generating random sequences. (time course ...)
