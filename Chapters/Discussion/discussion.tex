\chapter{Discussion}

The network dynamics of the brain and its sub-systems is inherently non-linear, which might provide additional layers of complexity required to achieve cognition in a purely material substrate. The neural networks of the brain are endowed with the capability to generate reliable activity patterns without any external stimulus. These internally self generated patterns put constraints on how the network behaves when external stimuli are presented. This was the underlying framework for this study. The experimental studies in the hippocampus have shown several phenomena that are internally generated. The hippocampus is also the center for spatial information processing. The discovery of place cells \cite{O'Keefe1971a} lead to the subsequent \emph{Cognitive Map hypothesis} \cite{Street}. Since then, a lot of studies have been conducted to understand the mechanisms for the formation of the spatial representation in the hippocampus. One of the predominant open questions has been - whether the spatial representation is innate or if the external stimuli is an absolutely necessary prerequisite for its formation. To solve this puzzle, it is necessary to first characterize the relevant dynamics in the hippocampus. Certain aspects of spatial processing in the hippocampus was the focus of this thesis in anticipation of discovery of interesting structure in the dynamics of the hippocampal sub-networks. 



\st{

Since 2 dimensional environments are closer to the spatial layout in the natural world, the place cells in 2D arena was analyzed in this study. The population activity in the CA3 and CA1 were analysed 


 The place cells in 2D arena was analyzed in this study.}\st{find evidence for the existence of attractors in the hippocampal network. 
search for the computational algorithms used in the hippocampus and its supporting structures for path integration and episodic memory. how well does the data agree with the existing models of path integration and remapping}


