\chapter{Discussion}
\lhead{Chapter 4. \emph{Discussion}} 

The network dynamics of the brain and its sub-systems is inherently non-linear, which might provide additional layers of complexity required to achieve cognition in a purely material substrate. The neural networks of the brain are endowed with the capability to generate reliable activity patterns without any external stimulus. These internally self generated patterns put constraints on how the network behaves when external stimuli are presented. This was the underlying framework for this study. The experimental studies in the hippocampus have shown several phenomena that are internally generated. The hippocampus is also the center for spatial information processing. The discovery of place cells \cite{O'Keefe1971a} lead to the subsequent \emph{Cognitive Map hypothesis} \cite{Street}. Since then, numerous studies have been conducted to understand the mechanisms for the formation of the spatial representation in the hippocampus. One of the predominant questions that remains unanswered is - whether the spatial representation is innate or if the external stimuli is an absolutely necessary prerequisite for its formation. To solve this puzzle, it is necessary to first characterize the relevant dynamics in the hippocampus. Certain aspects of spatial processing in the hippocampus was the focus of this thesis in anticipation of discovery of interesting structure in the dynamics of the hippocampal sub-networks. 

The place cells in this data set had multiple place fields in the square arena. Perhaps this is an indication that the current location of the animal is not represented in the activity of a single cell (ensemble encoding hypothesis). Rather, it is possible to conceive of a different encoding strategy, where the current location is encoded as a distributed activity in the population. Thus each location is now represented in a high dimensional neuronal space, and each location in the neural space is a vector whose components are the activities of cells that form the basis. This hypothesis, predicts irregularly spaced multiple place fields in large environments as a natural outcome of the encoding scheme. Place cells participating in several environments would not be an issue in this encoding approach, since all that is sufficient to have unique vectors for each location in different environments. Place cells are indeed found to be active in several environments\cite{Kubie1987}. Some experimental studies have been actually conducted to test the ensemble coding hypothesis \cite{Fenton2008}.  An alternative explanation could be that the multiple place fields observed could be result of the some spikes from several cells getting clustered into a single cluster i.e poor cluster isolation.

The analysis of the representation of a familiar environment show that it is stable across multiple recordings sessions. At the single cell level, analysis of the data in the same environment showed that the firing rates were reliable. The population activity activity was maintained at a value which fluctuated very little during periods of exploration, which is perhaps the biologically imposed homeostasis. For individual cells this would imply that when some of the cells increase their firing rate, other cells are forced to reduce their firing rates. 

In one dimensional tracks it has been shown that the theta phase of spike times reliably encodes the location of the animal within the place field (temporal encoding). \cite{O'Keefe1993, Skaggs1996c, Huxter2003}. A few experimental studies have reported temporal encoding in the open field \cite{Huxter2008a}. Pairwise co firing analysis was performed in order to probe how the position in physical space translated into the temporal activity in the neural space.The examination of pairwise spike CCGs did not offer much results. It was chiefly due issues of insufficient spatial occupancy of the animal and the multiple sub-fields. It has been shown theoretically that the stimulus space properties can be extracted form the activity of simulated groups of co-active cells \cite{Curto2008}. Along theses lines, the next question asked was whether the pair wise distances for the the place fields of cells active in distinct environments are preserved across environments with different topology. Since, the number of cells active in distinct environments were very few, it was not possible to evaluate this idea. \\

The activity during sleep offers an excellent vista of the internal activity in the hippocampal and cortical networks. The SWR during slow wave sleep are highly synchronous population activity. Some of the activity patterns generated during awake state are reactivated during sleep. The reactivated activity is implicated in the formation of memory trace \cite{Buzsaki1989}. The sleep sessions before and after exploratory sessions have ongoing internal activity. How this activity might be altered by the activity during exploration is an important issue to that has to be addressed. In addition, an equally crucial issue is how the internal activity biases the firing patterns during exploration. Answering these questions would be beneficial for solving the issue of the contributions from internal dynamics and external input in the producing the observed activity patterns.  \\

Initially the Bayesian decoding approach was employed with the aim of decoding the population events during sleep, the animals position from the place field spikes had very low accuracy in the 2D arena due to insufficient numbers of cells sampled from the population and low coverage of the environment. The decoded position showed discontinuous jumps. It has been shown that the mean decoding error that can achieved in the Bayesian framework is inversely proportional to the square root of the number cell (this is valid for sufficiently large number of cells) \cite{Zhang2013}. For the same reason the analysis of population events in the 2D environment did not yield any significant results. An additional issue is that of the inherent variability of the place cell firing. It has been experimentally demonstrated that the place cells have excessive firing variance than expected in the time domain even though they have reliable firing in the position domain \cite{Fenton1998}. This variance contributes to the reduction of decoding accuracy.  

A simpler measure of cofiring likelihood was employed to assess the pairwise changes in firing activity in sleep before and after exploratory sessions. The cofiring likelihood is expected to increase for cell pairs that have neighboring place fields. This is due to the fact that these cells would always be coactivated whenever the animal passes through the place fields resulting in strengthening the synaptic strength between the pair. (computing .... ) 
\st{The cofiring likelikhood decreased as expected with increasing distance between the place field peaks }\\

In linear tracks, a simpler template matching method was used instead of the decoding approach. This allows for examination of the cell sequences readily, albeit it has some shortcomings \cite{Tatsuno2006}.The population events were correlated with the templates constructed from the the cell sequences that fired when the animal traversed the linear track. There were $4.45 \% $ events in the pre-trial sleep that had significant correlation with the template and there were $ 2.24 \% $  events that correlated with the post-trial sleep. These significant events were detected at a predetermined threshold, it might lead to the false negatives for events with lower correlation values. So the difference between the cdfs were used and the significance of the differences were computed. Subsequently, the frequency of occurrence of the significant events was examined, but a small number of such events prevented further examination.\\ Surprisingly, in this data set, there was no enhancement of post-trial reactivation of sequences from preceding exploration in both in both CA3 and CA1 place cell populations. On the contrary, the post-trial sleep was reactivation frequency was significantly smaller, This requires further validation. Nevertheless if it is true; experience does not have as big an impact on the reactivation of sequences as usually presumed. Then the observed events might actually be dictated largely by internal dynamics. 

It was observed that the majority of cells (stats ....) that fired during population events were not active when the animal was running on the linear track. To test if these cells have reliable firing sequences, the sequences were correlated with each other and sampling distributions for these correlations were computed by generating random sequences \cite{Nadasdy1999}. (computing..... \st{The time course of the frequency of reliable sequences during sleep ...})\\

In summary, the single cell and population activity in the CA3 and CA1 were analysed. Major part of the work involved developing general MATLAB scripts for performing the data analysis on the large data sets. Due to the lack of statistically sufficient number of stable place fields with single place fields in the 2D arena, some of the questions posed could not be conclusively answered. The single cell activity is highly variable, but in a given environment; the population activity is maintained at a constant level during periods of exploration. The pairwise analysis suffered from insufficient numbers of cells. There was a small number of cell sequences activated in both the pre and post trial sleep sessions that were significantly correlated with the sequences during the trial. The quantification of change in the frequency of expression of cell sequences before and after exploration puts current views held on the importance of reactivation in the field under suspicion, further investigations are necessary to elucidate the importance of internal dynamics. 

\section*{Outlook}




