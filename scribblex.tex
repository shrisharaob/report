\documentclass{article}
\begin{document}
encoding - 

\section*{sequences}
A network-level coordination of spikes may occur due to a reverberation of the APs or a consistent external drive. The reverberation hypothesis was first proposed by Hebb (1949). According to the this hypothesis APs propagate in a circuit of interconnected cells as in a closed-loop. In a sparse-connectivity network, many closed-loops can be formed on a random basis. Recently, sparsely connected networks became a subject of intensive study (e.g. Amit, 1989). Attractor dynamics are likely to occur within sparsely connected network under certain conditions (Amit, 1989). The network dynamics will converge to a sequence of steady states i.e. to the limit cycle attractor which best supports the coincident input activity to specific cells. More precisely, only those feed-back loops will be stabilized which converge back to the already active cells. According to computer simulations, the conditions necessary for repetition to occur are as follows: (1) spontaneous activity of some cells, (2) fixed refractory periods and (3) feed-back loops in the network (Nadasdy, 1998). The anatomy of the hippocampus CA3 recurrent collateral system is consistent with a sparsely connected network with feed- back loops. In


\section*{place cells - refs}
Gothard 1996 - open field recordings \\
Muller 1994 - the firing rat of place cells are independent of head direction in the open field\\
similar to above : Markus 1995 - the directionality of the place cell depends on the trajectories traversed\\
Wilson MccNaughton 1993 - place cells recorded with tetrode shower fewer subfields\\
Save 1998 - place fields can be also associated with non-visual cues\\
Muller-Kubie 1987 and O'Keefe, Speakman 1987 - Place fields continued to persist after the visual land marks were removed \\
Pivo 1985 - removing some cues did not disrupt the place fields but removing all cues did\\
Quirk, Muller and Kubie 1990 - place fields persisted in the dark when the animals are placed with light and the light turned off.\\
O'keefe, Recce 1993 and Skaags 1996 - spike fired precesses with theta phase as the animal traverses theplace field, also occurs in 2D\\
Mehta, Barenes and Mc Naughton1997 - pfs shift backwards after repeated traversals along the same trajectory 
Place cell firing is also correlated with speed, direction, and turning angle  and stage of the task \\
Markus 1995 - pfs are task sensitive, pfs rapidly changed when the task changed to random foraging and search for food at the corners of a diamond\\
Skaags McNaughton 1996 - found significant correlated temporal bias during SWRs in the SWS after exploration on a linear track.\\
extra hippocampal cells have also been reported, but show different charecteristics the hippocampal place cells\\





\section*{remapping}
O'keefee and burgess 2005- place field firing could be superposition of grid cells, and remappig could be changes ib the offset of grid cells

\bibliographystyle{unsrt}
\bibliography{library.bib}
\end{document}